%! Author = saili
%! Date = 2019/8/26/0026

\section{安装与配置方案}\label{sec:安装方案}
\subsection{各平台安装方法最佳实践}
\subsubsection{Windows平台}
Windows下推荐采用TexLive\&VSCode\&SumatraPDF。

在\href{https://www.tug.org/texlive/windows.html}{TexLive}官网下载并安装TexLive,随后安装\href{https://www.sumatrapdfreader.org/free-pdf-reader.html}{Sumatra PDF}和\href{https://code.visualstudio.com}{VSCode}

安装好TexLive后,在命令控制台中输入
\begin{languagebox}[bash]
    latex -v
    xelatex -v        
\end{languagebox}
如果能够输出版本信息,证明Tex已经安装好了。

随后需要简单更改一下TexLive的配置,找到TexLive安装根目录下的\highunderline{texmf.cnf}文件,打开并添加如下的一行参数,其余配置维持不变:
\begin{languagebox}[bash]
    openout_any = a    
\end{languagebox}

如果不添加该参数,在添加参考文献时可能会报错。

随后,在VSCode中安装Tex扩展,快捷键\highunderline{Ctrl+Shift+X}打开扩展面板,查找并选择\highunderline{LateX Workshop}并安装:
\begin{figure}[H]
    \centering
    \includegraphics[width=0.8\columnwidth]{\figpath{lworkshop.png}}
    \caption{\LaTeX{}扩展}
    \label{fig:latex-vscode}
\end{figure}
随后需要在vscode的设置中配置编译命令,从而能够用快捷键迅速编译,配置主要是配置编译的工具和食谱(recipes,即工具的联合使用),官网提供了很详细的说明\href{https://marketplace.visualstudio.com/items?itemName=James-Yu.latex-workshop}{文档},以及\href{https://zhuanlan.zhihu.com/p/38178015}{这篇博客}也提供了比较详细的配置方法。

如果嫌麻烦,目前的vscode配置支持和其他用户分享GistID来共享配置,可以使用我的配置对应的GistID(可能仍需要配置PDF阅读器的位置):

\begin{quotation}
    999ba45b6ca3350048541fd7f72ef07a
\end{quotation}

\begin{figure}[H]
    \centering
    \includegraphics[width=0.8\textwidth]{\figpath{gistid.png}}
    \caption{设置GistID共享VSCode配置}
    \label{fig:gist-id-ex}
\end{figure}

安装好后,任意新建一个tex后缀的文档,输入以下内容(注意用半角标点符号,以及去掉前面的注释符号\%\footnote{因为未知原因,代码环境无法插入该语句,还未解决}):
\begin{verbatim}
    % \documentclass{article}
    % \begin{document}
    %     hello world
    % \end{document}
\end{verbatim}
正常编译\marginnote{如果使用了之前的配置ID,那么快捷键是Shift+F10}后如果能够生成一个pdf文档,那么就说明安装完成了。
\begin{figure}[H]
   \centering
   \includegraphics[width=0.5\textwidth]{\figpath{texplug.png}}
   \caption{VSCode编译选项示意}
   \label{fig:texplug.png}
\end{figure}
\subsection{中文支持}
\LaTeX{}有比较多的中文支持方法,目前最好用的是\highunderline{ctex}包。以及虽然ctex提供了一些文档类,但是还是推荐用添加包的方式来使用,这样会比较容易兼容纯英文方案。

\begin{texcode}
    \usepackage[UTF8,fontset=windowsnew,heading=true]{ctex} % 使用包的方式添加时推荐使用的参数
\end{texcode}

之后默认情况下不需要其他配置,就可以完美的支持中文了,但是注意,需要使用\highunderline{XeLaTeX}编译。
\subsubsection{Linux平台}
不熟悉,暂略
\subsubsection{MAC平台}
不熟悉,暂略

\subsection{常用包}
\LaTeX{}由于其开放的环境,有成千上万的包提供用户使用,在一般情况下,比较常用的包和配套的参数如下所示:
\begin{texbreakcode}
    \usepackage{amsmath}%提供数学公式支持
    \usepackage{graphics}%用于添加图片
    \usepackage{graphicx}%加强插图命令
    \usepackage{subfigure}%用于添加子图
    \usepackage{wrapfig}%提供图片环绕风格支持
    \graphicspath{{./}{./contents/}{./contents/fig/}}%设置图片可能存在的路径
    \newcommand{\figpath}[1]{contents/fig/#1} %设置图片路径

    \usepackage{fontspec}%用于配置字体
    \usepackage[table]{xcolor}%用于各种颜色环境
    \usepackage{enumitem}%用于定制list和enum
    \usepackage{float}%用于控制Float环境,添加H参数(强制放在Here)
    \usepackage[colorlinks,
                linkcolor=black,
                urlcolor=blue,
                anchorcolor=blue,
                citecolor=green]{hyperref}%用于超链接,另外添加该包目录会自动添加引用。

    \usepackage[most]{tcolorbox}%用于添加各种边框支持,most参数表示将全部命令支持都添加进来
    \usepackage[cache=true,outputdir=./out]{minted}%如果不保留临时文件就设置cache=false,如果输出设置了其他目录那么outputdir参数也有手动指定,否则会报错。
    \tcbuselibrary{minted}%加载tcolorbox的代码风格

    \usepackage[a4paper,left=4cm,right=4cm,top=3cm,bottom=1cm]{geometry}%用于控制版式
    \usepackage{appendix}%用于控制附加文件

    
    \usepackage[UTF8,fontset=windowsnew,heading=true]{ctex} %用于提供中文支持
    \ctexset{
        section = {
        number = 第\chinese{section}章,
        format = \zihao{3}\bfseries,
        },
        subsection = {
        number = \arabic{section}.\arabic{subsection},
        format = \Large\bfseries
        },
        subsubsection = {
        number = \arabic{section}.\arabic{subsection}.\arabic{subsubsection},
        format = \Large\bfseries,
        },
    }
    \usepackage{multirow} %用于表格环境多行多列的支持
    \usepackage{pdfpages} %用于插入pdf页
    \usepackage{ulem} %用于好看的下划线、波浪线等修饰
    \usepackage{fixltx2e} %用于文本环境的下标

    \usepackage{adjustbox} %用于调整"盒子"位置
    \usepackage{longtable}% 用于提供长表格支持        
    \usepackage{nameref} % 用于在引用时附加描述而不只是计数值
\end{texbreakcode}

\subsection{\LaTeX{}内置命令}
    \LaTeX{}中除了编译器相关的命令外,还提供了许多别的非常方便的命令,列举如下:
    \subsubsection{查看帮助文档}
    在熟练\LaTeX{}中的操作后,如果需要查看某个宏包的文档,可以尝试在命令行输入:
    \begin{languagebox}[bash]
        texdoc xxx
    \end{languagebox}
    
    \subsubsection{其他命令}
    \begin{center}
        % \newlength\tablewidth % if haven't define the length 'tablewidth'
        \setlength\tablewidth{\dimexpr (\textwidth -4\tabcolsep)}
        \begin{table}[H]
            \begin{tabular}{|p{0.50\tablewidth}<{\centering}|p{0.50\tablewidth}<{\centering}|}
                \hline
                命令名&备注\\
                \hline
                bibtex&参考文献支持。\\
                \hline
                makeindex,xindy&索引支持。\\
                \hline
                dvips&将DVI转换为PostScript\\
                \hline
                xdviX&WindowSystem下的DVI阅读器。\\
                \hline
                dviconcat,dviselect&从DVI文件中复制和粘贴页面。\\
                \hline
                dvipdfmx&将DVI转换为PDF,是(前面提到过的)pdfTEX的一套替换方案。\\
                \hline
                psselect,psnup,...&PostScript实用程序。\\
                \hline
                pdfjam,pdfjoin,...&PDF实用程序。\\
                \hline
                context,mtxrun&ConTEXt和PDF处理工具。\\
                \hline
                htlatex,...tex4ht&(LA)TEX到HTML(还有XML等其他格式)的转换器\\
                \hline
            \end{tabular}
            \caption{TexLive中的常用程序}
        \end{table}
    \end{center}

\subsection{快速入门}\label{sub:快速入门}
    在按上面的方式安装好并完成编译测试后,可以按以下顺序查看,:
    \begin{itemize}
        \item \Ref{sec:comm-envi}
        \item \Ref{sec:页面处理}
        \item \Ref{sec:标题和目录}
        \item \Ref{sec:大小与间距}
        \item \Ref{sec:文字样式}
        \item \Ref{sec:段落样式}
        \item \Ref{sec:列表}
        \item \Ref{sec:graphics}
        \item \Ref{sec:table}
        \item \Ref{sec:公式}
        \item \Ref{sec:Float}
        \item \Ref{sec:注释与引用}
        \item \Ref{sec:color}
    \end{itemize}
    