\section{注释与引用}\label{sec:注释与引用}
    \subsection{脚注}
    \subsubsection{基本使用}
    脚注使用\highunderline{\textbackslash{}footnote{内容}}命令:
    \begin{texshow}
        被脚注的文本\footnote{脚注内容}
    \end{texshow}
    \subsubsection{更改标记风格}
    需要重新定义\highunderline{\textbackslash{}thefootnote}来实现更改标记风格:

    \begin{texcode}
        被脚注的文本\footnote{脚注内容}\\
        \renewcommand{\thefootnote}{\roman{footnote}}
        被脚注的文本\footnote{脚注内容}
    \end{texcode}

    
    被脚注的文本\footnote{脚注内容}\\\renewcommand{\thefootnote}{\roman{footnote}}
    被脚注的文本\footnote{脚注内容}

    关于各种风格的数字显示方式,查看
    \subsubsection{重新计数脚注标记}
    如果只是要指定某个脚注的下标,可以直接通过添加参数的方式:
    \begin{texshow}
        被脚注的文本\footnote[12]{脚注内容}
    \end{texshow}


    脚注标记的计数也是用\LaTeX{}中的计数器来计数的,名字是\highunderline{footnote},因此要重新计数,只需要重置计数器\footnote{关于计数器,可以参考\Ref{subsub:counter}}即可:
    \begin{texcode}
        被脚注的文本\footnote{脚注内容}\\
        \setcounter{footnote}{5}
        被脚注的文本\footnote{脚注内容}
    \end{texcode}

    被脚注的文本\footnote{脚注内容}\\
        \setcounter{footnote}{5}
    被脚注的文本\footnote{脚注内容}

    \subsection{边注}

    \subsubsection{基本使用}
    边注是指在书两侧的注释,其在某种程度上比脚注更方便一些,因为边注直接显示在页面的一边\marginnote{就像这样。},读者不需要将视线移动太多就能看到相应的注释。

    最简单的边注,使用\highunderline{\textbackslash{}marginpar[left]\{right\}}命令。\marginpar{最简单的边注}如果是单侧布局(即不分奇偶页),那么边注仅显示在右侧,如果是双侧布局,则会显示在相应的外侧。如果是双栏布局,那么就会显示在靠近的一侧。

    \subsubsection{marginnote宏包}
    内置的这两种命令对于一些环境的支持不是很好,比如显示在公式的一侧,或者是显示在脚注的一侧,如果原生命令无法满足需求,那么就可以使用扩展宏包\highunderline{marginnote}。

    首先使用marginnote宏包后,\highunderline{geometry}就增加了可以控制边注大小和间隔的参数,如下:
    \begin{texcode}
        \usepackage[top=Bcm, bottom=Hcm, outer=Ccm, inner=Acm, 
            heightrounded, 
            marginparwidth=Ecm, marginparsep=Dcm]{geometry}
    \end{texcode}

    在设置好边距后,如果要添加边注,使用\highunderline{\textbackslash{}marginnote}命令:
    \begin{texshow}
        这里是一段难懂的命令\marginnote{这是一段解释的文字。}。
    \end{texshow}
    
    上面使用的命令如果更换为marginpar,在演示环境中会报错,这也是为什么推荐使用marginnote的原因。

    \subsection{标签}\label{sub:标签}
    \LaTeX{}支持在任意位置添加标签,从而可以很方便的建立链接,如\marginnote{从实现中可以看到标签的定义支持中文和一些标点,另外最好遵循某种规范,方便寻找}:
    \begin{texshow}
        如何实现一段标签可以查看该示例:\label{ex:标签示例}
    \end{texshow}
    要引用只需要使用内置命令\highunderline{\textbackslash{}ref}即可:
    \begin{texshow}
        如何使用引用可以参考\ref{ex:标签示例}节的介绍。
    \end{texshow}
    \begin{texshow}
        我们建议初学者浏览\pageref{sub:快速入门}页来查看如何入门\LaTeX{}
    \end{texshow}

    注意带有引用的位置均需要编译两次来完成,第一次编译只会形成问号,这是正常的。
    \subsubsection{其他的引用需求}
    ref命令只会显示相应位置的节号,而label命令实际上只是建立一种链接关系,因此,如果要更改引用部分的文字,我们可以使用\highunderline{hyperref}宏包中的\highunderline{\textbackslash{}hyperref}命令:
    \begin{texshow}
        % \usepackage{hyperref}
        这里有一个\hyperref[ex:标签示例]{标签示例}
    \end{texshow}
    注意如果链接到一个URL,那么使用的是\highunderline{\textbackslash{}href}命令,而且两者的参数填写方法也不一样:
    \begin{texshow}
        关于hyperref宏包的具体参数,可以参考其文档,或者查看
        \href{https://en.wikibooks.org/wiki/LaTeX/Hyperlinks}
        {Wiki-Hyperlinks}
    \end{texshow}

    另外,\highunderline{nameref}宏包提供了显示章节号的命令\highunderline{\textbackslash{}nameref}:
    \begin{texshow}
        % \usepackage{nameref}
        在\nameref{ex:标签示例}有一个示例
    \end{texshow}

    \subsection{参考文献}
    参考文献在\LaTeX{}中有多种管理方式,并且均可以被很方便的引用。\footnote{另外文献管理推荐使用\href{http://zotero.org}{zotero}}

    \subsubsection{直接引入}
    如果参考文献的数量比较少,不需要管理,那么可以直接使用\LaTeX{}的内置环境\highunderline{thebibliography},该环境会直接在相应位置生成参考文献列表:
    \begin{texshow}
        \begin{thebibliography}{9}
            \bibitem{lamport99}
              Leslie Lamport,
              \textit{\LaTeX: a document preparation system},
              Addison Wesley, Massachusetts,
              2nd edition,
              1994.
            \bibitem{SMOalgo}
                Platt,John C,
                \textit{Sequential Minimal Optimization: A Fast Algorithm for Training Support Vector Machines}
        \end{thebibliography}

        \begin{thebibliography}{1}
            \bibitem{tmc}
                Petersen,Pedersen,《The Matrix Cookbook》
        \end{thebibliography}
        
    \end{texshow}

    上述的代码中,环境后的参数为最大标签数(必填项),但是要注意的是,其中的参数不是数字大小,而是数字的位数,如果是一位数字,那么最大标签数是9;如果是两位,那么最大标签数是99(即使你所填的数字是59)。

    每一个参考文献,都用一个\highunderline{bibitem}来标识,\highunderline{bibitem}之后到下一个\highunderline{bibitem}之前的内容均属于该bibitem。

    如果要引用其中的其中的某一个参考文献,使用\highunderline{cite}命令\footnote{标签能够具有颜色和链接属性是因为hyperref库}:
    \begin{texshow}
        在这时候,有人发现了求解SVM的更快、更有效率的算法\cite{SMOalgo}
    \end{texshow}

    使用这种方法,跟目录等交叉引用项一样,需要用相应的编译器编译两次。

    \subsubsection{bibtex引入}

    直接引入虽然方便,但是当参考文献很多的时候,会很不方便,因为一般情况下,没有引用到的文献是不需要添加在参考文献中的,而且不同的组织,其参考文献如何排序都是有标准的,这些如果我们如果使用直接引入的方法,就必须全都考虑,就会很麻烦。由此,我们可以使用\highunderline{bibtex}来帮助我们管理参考文献和辅助引用。

    \highunderline{bibtex}基于\highunderline{.bib}文件,该文件可以被当做是数据库,可以把我们使用过的所有的参考文献都按照bibtex的格式放进去。其中,每一篇参考文献的格式均如下所示:
    \begin{texcode}
        @article{AbedonHymanThomas2003,
            author = "Abedon, S. T. and Hyman, P. and Thomas, C.",
            year = "2003",
            title = "Experimental examination of bacteriophage latent-period evolution as a response to bacterial availability",
            journal = "Applied and Environmental Microbiology",
            pages = "7499--7506",
            ...
        }
    \end{texcode}
    其中第一个参数是该引用的标识,而针对不同的引用(论文、期刊、书籍、网页...等),只需要按需填写不同的参数即可,如引用网页的示例格式如下:
    \begin{texcode}
        @misc{website:fermentas-lambda,
            author = "Fermentas Inc.",
            title = "Phage Lambda: description \& restriction map",
            month = "November",
            year = "2008",
            url = "http://www.fermentas.com/techinfo/nucleicacids/maplambda.htm"
        }
    \end{texcode}

    上述的内容可以全部存放到一个.bib文件中,也可以分门别类存放到多个.bib文件中,均可,只需要最后在使用的时候,在需要显式参考文献列表的位置添加如下的两行代码即可:
    \begin{texcode}
        \bibliographystyle{plain}
        \bibliography{bf1,bf2,...,bf3}
    \end{texcode}
    其中\highunderline{bibliographystyle}命令是表明参考文献展示的格式,包括排序方式,展示规范等,共有8种,分别为:
    \begin{center}
        % \newlength\tablewidth % if haven't define the length 'tablewidth'
        \setlength\tablewidth{\dimexpr (\textwidth -4\tabcolsep)}
        \begin{table}[H]
            \begin{tabular}{|p{0.3\tablewidth}<{\centering}|p{0.7\tablewidth}<{\centering}|}
                \hline
                plain&按字母的顺序排列,比较次序为作者、年度和标题.\\
                \hline
                unsrt&样式同plain,只是按照引用的先后排序.\\
                \hline
                alpha&用作者名首字母+年份后两位作标号,以字母顺序排序.\\
                \hline
                abbrv&类似plain,将月份全拼改为缩写,更显紧凑.\\
                \hline
                ieeetr&国际电气电子工程师协会期刊样式.\\
                \hline
                acm&美国计算机学会期刊样式.\\
                \hline
                siam&美国工业和应用数学学会期刊样式.\\
                \hline
                apalike&美国心理学学会期刊样式.\\
                \hline
            \end{tabular}
            \caption{参考文献的显示格式}
        \end{table}
    \end{center}

    第二行\highunderline{bibliography}命令表示在这里显示参考文献列表,参数填写参考文献都在哪些文件里,可多不可少。在具体编译完成后,相应位置只会显示文章中引用过的参考文献,.bib文件中其他的参考文献则不会显示。

    这种方式在文献的管理上确实比直接引用要好很多,但是也有一个非常麻烦的地方,那就是要用这种方式编译文件,需要编译四次,以使用xeLaTeX,编译main.tex为例,四次编译依次为:
    \begin{languagebox}[bash]
        xelatex main.tex
        bibtex main.aux
        xelatex main.tex
        xelatex main.tex
    \end{languagebox}
    注意第二次的参数是.aux文件而不是.tex文件,是在第一次编译后生成的,bibtex需要使用.aux文件来获取.tex文件中引用了哪些文献,从而生成相应的.bbl文件。.bbl文件的内容其实就是直接由bibtex从.bib文件中排好格式并且挑选出来相应文献后的\highunderline{thebibliography}环境,随后两次编译则是编译交叉引用的正常步骤。

    回头来看bibtex和.bib,它们实质上是依托于tex系统而形成的一套文献管理软件,其通过借助.tex文件编译后形成的引用信息来生成\LaTeX{}支持的参考文献环境。

