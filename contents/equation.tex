\section{公式环境}\label{sec:公式}
    行内公式、行间公式、下标显示、label添加、公式编辑工具

    公式环境也叫做数学环境,是用来处理各种各样的数学符号的,并且支持多种展示属性,一般情况下,使用\highunderline{amsmath}包即可
    

    \subsection{行内公式}
    行内公式可以使用\highunderline{math}环境,\verb|\(..\)|或者\verb|$...$|三种方法:

    \begin{texshow}
        % \usepackage{amsmath}
        正常文本嵌套行内公式A:\begin{math}f_i(x)=a+b\end{math}。\\
        正常文本嵌套行内公式B:\(f^2(x)=a*b\)。\\
        正常文本嵌套行内公式C:$g(l;\theta)=a-b$。
    \end{texshow}

    \subsubsection{数学环境中的普通文本}
    如果要在公式内添加解释文本,英文可以直接支持,但是会转换成公式默认字体(itali斜体),并且不支持中文输入,因此如果要在公式环境内添加解释,需要使用\highunderline{amsmath}包中的\highunderline{\textbackslash{}text}命令:
    \begin{texshow}
        $g(l;\theta)=a-b,\text{正常文本 normal text},正常文本 normal text$
    \end{texshow}

    \subsection{行间公式}
    首先一定要注意,\textbf{行间公式不能有空行}!所以每一条公式必须紧靠着。如果要换行,需要使用\verb|\\|声明(仅在\textbf{多行公式}环境中有效)。
    
    \subsubsection{单行行间公式}
    单行行间公式不支持换行符(会忽略换行符)。

    如果不需要对行间公式进行编号,可以直接使用\verb|\[...\]|或\verb|$$...$$|
    \begin{texshow}
        公式前文本
        \[
            x = a+b;\\
            y = c+d;
        \]
        公式后文本
    \end{texshow}

    如果需要对一整个行间公式进行编号,可以使用\highunderline{equation}环境:
    \begin{texshow}
        \begin{equation}
            x = a+b;\\
            y = c+d;
        \end{equation}
    \end{texshow}

    \subsubsection{多行行间公式}
    如果要分段定义编号,可以使用\highunderline{eqnarray}环境,会对每一部分单独编号,如果不需要编号,在当前行添加\highunderline{\textbackslash{}nonumber}:
    \begin{texshow}
        \begin{eqnarray}
            x = a+b;\\
            y = c+d;\nonumber\\
            z = e+f;
        \end{eqnarray}
    \end{texshow}
    如果要对多个公式单独编号,但有对其中几个公式完成组编号,可以在公式中嵌套\highunderline{split}环境(注意在环境结束后添加换行符):
    \begin{texshow}
        \begin{eqnarray}
            \begin{split}
                a = 1;\\
                b = 2;\\
                c = 3;\\
            \end{split}\\
            x = a+b;\\
            y = c+d;\nonumber\\
            z = e+f;
        \end{eqnarray}
    \end{texshow}
    如果要重新开始编号或从指定数字开始编号,使用\highunderline{\textbackslash{}setcounter\{equation\}\{1\}}方法。

    \begin{texshow}
        \begin{eqnarray}
            \setcounter{equation}{2}
            \begin{split}
                a = 1;\\
                b = 2;\\
                c = 3;\\
            \end{split}\\
            \setcounter{equation}{7}
            x = a+b;\\
            y = c+d;\\
            z = e+f;
        \end{eqnarray}
    \end{texshow}

    如果需要对齐,则使用\highunderline{align}环境,并在要对齐的地方使用\&声明:
    \begin{texshow}
        \begin{align}
            &f(x) = a+b+c;\\
            g(x) = a;&
        \end{align}
    \end{texshow}
    同时,在\highunderline{eqnarray}环境中可以直接使用\highunderline{align}的方式进行对齐

    \subsection{公式编辑工具推荐}
    \subsubsection{Mathpix}
    网站链接:\href{https://mathpix.com/}{Mathpix}。

    如果有图片类型的公式,可以用该软件截图生成\LaTeX{}公式,多平台支持,且准确率很高,即使多行公式也能有很高的准确率,同时免费用户每天有50次截图的限额,非常划算。
    \begin{figure}[H]
       \centering
       \includegraphics[width=0.8\columnwidth]{\figpath{mathpix.png}}
       \caption{Mathpix官网示例}
       \label{fig:mathpix}
    \end{figure}


    \subsubsection{在线\LaTeX{}公式编辑器}
    网站链接:\href{https://www.codecogs.com/latex/eqneditor.php}{eqneditor}

    网站丑了点,但是能够对公式进行所见即所得的编辑,比较友好,目前暂时没有找到比它更好的替代方案,不过如果配置时使用的是Tex专门的IDE,其公式编辑应该也比较方便(不过可能不支持所见即所得)