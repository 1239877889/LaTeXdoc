\documentclass{ctexart}
\usepackage{tikz,lipsum,lmodern}
\usepackage[most]{tcolorbox}
\usepackage{lipsum}
\definecolor{bidentitlebg}{RGB}{158,59,255}
\usepackage{mtpro2}
\usepackage{url}
\usepackage{varwidth}
\everymath{\displaystyle}
\usepackage{geometry}  
\tcbuselibrary{skins,breakable,theorems}
\geometry{left=3cm,right=2.5cm,top=2.5cm,bottom=2.5cm}  
%\usepackage[top=1cm,bottom=1cm,left=1cm,right=1mm]{geometry}  
\usepackage{listings} 
\lstset{numbers=left,breaklines,
numberstyle= \tiny,keywordstyle= \color{ blue!70},commentstyle=\color{red!50!green!50!blue!50},
frame=shadowbox, rulesepcolor= \color{ red!20!green!20!blue!20},
escapeinside=``} 
\newtcolorbox{Mybox}[2][]
{colback = red!5!white, colframe = red!75!black, fonttitle = \bfseries,
	colbacktitle = red!85!black, enhanced,
	attach boxed title to top center={yshift=-2mm},
	title=#2,#1}

\title{tcolorbox包的学习使用}
\author{八一}
\date{\today}
\begin{document}
\maketitle
\begin{abstract}
	学习tcolorbox 宏包使用是在我们想要更好地向他人展示一个作品时,利用这个包可以做出很多你想要的东西.和大多数宏包一样,tcolorbox 被 MiKTeX 和 TeX Live 都收录其中,使用起来非常简单,只需要你在导言在使用 usepackage 命令调用这个宏包。
\begin{lstlisting}[language={TeX}]
\usepackage{tcolorbox}
\end{lstlisting}

在调用宏包时,你可以通过可选参数的需要取加载其程序库,在调用 tcolorbox 之后,显式地使用 tcbuselibrary 来调用 tcolorbox 提供的丰富程序库。比如,下面的代码调用了 skins, breakable, theorems 三个程序库。
\begin{lstlisting}[language={TeX}]
\usepackage{tcolorbox}
\tcbuselibrary{skins, breakable, listings,theorems}
\end{lstlisting}

使用 tcbuselibrary 命令,就是调用了一个个这样的文件。以下是常见可用的 tcolorbox 程序库,比如breakable实现自动分页的文本框;skins提供丰富的文本框样式;listings/minted: 用以和对应的宏包联用,实现好看的代码清单;theorems会自动加载 amsmath 宏包来实现定理类的环境.

而tcolorbox 宏包提供了与宏包同名的环境,是整个宏包的基础环境,用于生成段落间的文本框,且还提供了 tcbox 命令,用于生成行内的文本框。
\begin{lstlisting}[language={TeX}]
\begin{tcolorbox}[⟨options⟩]
⟨environment content⟩
\end{tcolorbox}
\tcbox[⟨options⟩]{⟨box content⟩}
\end{lstlisting}

那么接下来我带大家一起学习下有趣的tcolorbox包的使用.
\end{abstract}
\section{基础阶段}	
\subsection{基础盒子}
\begin{lstlisting}[language={TeX}]
\begin{tcolorbox}[colback=red!5!white,colframe=red!75!black]
	My box.
\end{tcolorbox}
\end{lstlisting}
	\begin{tcolorbox}[colback=red!5!white,colframe=red!75!black]
		My box.
	\end{tcolorbox}
\subsection{加标题 title 的参数}
\begin{lstlisting}[language={TeX}]
	\begin{tcolorbox}[colback=blue!5!white,colframe=blue!75!black,title=`给框加标题`]
	My box with my title.
	\end{tcolorbox}
\end{lstlisting}
	\begin{tcolorbox}[colback=blue!5!white,colframe=blue!75!black,title=`给框加标题`]
		My box with my title.
	\end{tcolorbox}

\subsection{tcblower加虚线}
\begin{lstlisting}[language={TeX}]
\begin{tcolorbox}[colback=green!5!white,colframe=green!75!black]
Upper part of my box.
\tcblower
Lower part of my box.
\end{tcolorbox}
\end{lstlisting}
\begin{tcolorbox}[colback=green!5!white,colframe=green!75!black]
	Upper part of my box.
	\tcblower
	Lower part of my box.
\end{tcolorbox}

\subsection{标题与分割}
\begin{lstlisting}[language={TeX}]
\begin{tcolorbox}[colback=yellow!5!white,colframe=yellow!50!black,
colbacktitle=yellow!75!black,title=My title]
I can do this also with a title.
\tcblower
Lower part of my box.
\end{tcolorbox}
\end{lstlisting}
\begin{tcolorbox}[colback=yellow!5!white,colframe=yellow!50!black,
	colbacktitle=yellow!75!black,title=My title]
	I can do this also with a title.
	\tcblower
	Lower part of my box.
\end{tcolorbox}

\subsection{添加savelowerto选项}
\begin{lstlisting}[language={TeX}]
\begin{tcolorbox}[colback=yellow!10!white,colframe=red!75!black,lowerbox=invisible,
savelowerto=\jobname_ex.tex]
Now, we play hide and seek. Where is the lower part?
\tcblower
I'm invisible until you find me.
\end{tcolorbox}
\end{lstlisting}
\begin{tcolorbox}[colback=yellow!10!white,colframe=red!75!black,lowerbox=invisible,
	savelowerto=\jobname_ex.tex]
	Now, we play hide and seek. Where is the lower part?
	\tcblower
	I'm invisible until you find me.
\end{tcolorbox}
\subsection{定制的段落样式}
\begin{lstlisting}[language={TeX}]
\newtcolorbox{bidentidad}[1][]{
enhanced,
skin=enhancedlast jigsaw,
attach boxed title to top left={xshift=-4mm,yshift=-0.5mm},
fonttitle=\bfseries\sffamily,
colbacktitle=blue!45,
colframe=red!50!black,
interior style={
top color=blue!10,
bottom color=red!10
},
boxed title style={
empty,
arc=0pt,
outer arc=0pt,
boxrule=0pt
},
underlay boxed title={
\fill[blue!45!white]
(title.north west) --
(title.north east) --
+(\tcboxedtitleheight-1mm,-\tcboxedtitleheight+1mm) --
([xshift=4mm,yshift=0.5mm]frame.north east) --
+(0mm,-1mm) --
(title.south west) -- cycle;
\fill[blue!45!white!50!black]
([yshift=-0.5mm]frame.north west) --
+(-0.4,0) --
+(0,-0.3) -- cycle;
\fill[blue!45!white!50!black]
([yshift=-0.5mm]frame.north east) --
+(0,-0.3) --
+(0.4,0) -- cycle;
},
title={Identidades},
#1
}
\begin{bidentidad}
\lipsum[4]
\end{bidentidad}
\end{lstlisting}
\newtcolorbox{bidentidad}[1][]{
	enhanced,
	skin=enhancedlast jigsaw,
	attach boxed title to top left={xshift=-4mm,yshift=-0.5mm},
	fonttitle=\bfseries\sffamily,
	colbacktitle=blue!45,
	colframe=red!50!black,
	interior style={
		top color=blue!10,
		bottom color=red!10
	},
	boxed title style={
		empty,
		arc=0pt,
		outer arc=0pt,
		boxrule=0pt
	},
	underlay boxed title={
		\fill[blue!45!white]
		(title.north west) --
		(title.north east) --
		+(\tcboxedtitleheight-1mm,-\tcboxedtitleheight+1mm) --
		([xshift=4mm,yshift=0.5mm]frame.north east) --
		+(0mm,-1mm) --
		(title.south west) -- cycle;
		\fill[blue!45!white!50!black]
		([yshift=-0.5mm]frame.north west) --
		+(-0.4,0) --
		+(0,-0.3) -- cycle;
		\fill[blue!45!white!50!black]
		([yshift=-0.5mm]frame.north east) --
		+(0,-0.3) --
		+(0.4,0) -- cycle;
	},
	title={Identidades},
	#1
}
\begin{bidentidad}
	\lipsum[4]
\end{bidentidad}
\subsection{在盒子添加.tex源文件}
\begin{lstlisting}[language={TeX}]
\begin{tcolorbox}[colback=yellow!10!white,colframe=red!75!black,title=试题解答]

\newcommand\wqy[1]{\CJKfontspec{文泉驿微米黑}#1}
\newcommand\fzxk[1]{\CJKfontspec{方正行楷简体}#1}
\newcommand\hwcy[1]{\CJKfontspec{华文彩云}#1}
\usetikzlibrary{patterns,decorations.text}
\usetikzlibrary{decorations.text}
\newcommand\fzxkjt[1]{\CJKfontspec{方正行楷简体}#1} 
\setCJKfamilyfont{kd}{华文行楷}

%%%%%%%%%%%%%%%%%%%%%%%%%%%%%%%%%%%%%%%%%%%%%%

\newcommand{\chaoda}{\fontsize{55pt}{\baselineskip}\selectfont}
\newcommand{\chuhao}{\fontsize{42pt}{\baselineskip}\selectfont}     % 字号设置
\newcommand{\xiaochuhao}{\fontsize{36pt}{\baselineskip}\selectfont} % 字号设置
\newcommand{\yihao}{\fontsize{28pt}{\baselineskip}\selectfont}      % 字号设置
\newcommand{\erhao}{\fontsize{21pt}{\baselineskip}\selectfont}      % 字号设置
\newcommand{\xiaoerhao}{\fontsize{18pt}{\baselineskip}\selectfont}  % 字号设置
\newcommand{\sanhao}{\fontsize{15.75pt}{\baselineskip}\selectfont}  % 字号设置
\newcommand{\xiaosanhao}{\fontsize{15pt}{\baselineskip}\selectfont} % 字号设置
\newcommand{\sihao}{\fontsize{14pt}{\baselineskip}\selectfont}      % 字号设置
\newcommand{\xiaosihao}{\fontsize{12pt}{14pt}\selectfont}           % 字号设置
\newcommand{\wuhao}{\fontsize{10.5pt}{12.6pt}\selectfont}           % 字号设置
\newcommand{\xiaowuhao}{\fontsize{9pt}{11pt}{\baselineskip}\selectfont}   % 字号设置
\newcommand{\liuhao}{\fontsize{7.875pt}{\baselineskip}\selectfont}  % 字号设置
\newcommand{\qihao}{\fontsize{5.25pt}{\baselineskip}\selectfont}    % 字号设置
%%%%%%%%%% 定理类环境的定义 %%%%%%%%%%
%% 必须在导入中文环境之后
\newtheorem{example}{Example}             %
\newtheorem{algorithm}{算法}
\newtheorem{theorem}{定理} 
\newtheorem{definition}{定义}
\newtheorem{axiom}{公理}
\newtheorem{property}{性质}
\newtheorem{proposition}{命题}
\newtheorem{lemma}{引理}
\newtheorem{corollary}{推论}
\newtheorem{remark}{注解}
\newtheorem{condition}{条件}
\newtheorem{conclusion}{结论}
\newtheorem{assumption}{假设}
\newtheorem{proof}{证明}
\newtheorem{solution}{解}
\newtcolorbox{smybox}[2][]{enhanced,breakable,
	before skip=2mm,after skip=2mm,
	colback=black!5,colframe=black!50,boxrule=0.2mm,
	attach boxed title to top left={xshift=0.5cm,yshift*=1mm-\tcboxedtitleheight},
	varwidth boxed title*=-0.5cm,
	boxed title style={frame code={
			\path[fill=tcbcol@back!30!black]
			([yshift=-1mm,xshift=-1mm]frame.north west)
			arc[start angle=0,end angle=180,radius=1mm]
			([yshift=-1mm,xshift=1mm]frame.north east)
			arc[start angle=180,end angle=0,radius=1mm];
			\path[left color=tcbcol@back!60!black,right color=tcbcol@back!60!black,
			middle color=tcbcol@back!80!black]
			([xshift=-2mm]frame.north west) -- ([xshift=2mm]frame.north east)
			[rounded corners=1mm]-- ([xshift=1mm,yshift=-1mm]frame.north east)
			-- (frame.south east) -- (frame.south west)
			-- ([xshift=-1mm,yshift=-1mm]frame.north west)
			[sharp corners]-- cycle;
		},interior engine=empty,
	},
fonttitle=\bfseries,
title={#2},#1}

\begin{center}
	{\erhao \CJKfamily{kd}{2019年南开大学数学分析试题解答} }\\
	\vspace{0.2cm}
	{\sihao \textbf{满分:150分,考试时间:150分钟} }\\
	\vspace{0.3cm}
\end{center}

\begin{smybox}[colbacktitle=green!25!black!10!white]{第8题}
	已知$\alpha,\beta$均为正实数,且$\max\left\{\alpha ,\beta\right\}>1$,试证:
	\[
	\lim_{x\rightarrow +\infty}\int_1^x{\frac{1}{x^{\alpha}+t^{\beta}}}\mathrm{d}t=0
	\]
\begin{proof}由题设易知有
	\begin{enumerate} 
		\item 当$\alpha>1$时,有
		\[
		0 \leq \int_{1}^{x} \frac{\mathrm{d}t t}{x^{\alpha}+t^{\beta}} \leq \int_{1}^{x} \frac{1}{x^{\alpha}} \mathrm{d}tt=\frac{x-1}{x^{\alpha}} \rightarrow 0(x \rightarrow \infty)
		\]
		\item 当$\beta>1,0<\alpha \leq 1$时,且$\left(x\rightarrow\infty\right)$有
		\[
		\int_1^x{\frac{\mathrm{d}t}{x^{\alpha}+t^{\beta}}}=x^{-\alpha +\frac{\alpha}{\beta}}\int_1^x{\frac{\mathrm{d}\left(x^{-\frac{\alpha}{\beta}}t\right)}{1+\left(x^{-\frac{\alpha}{\beta}}t\right)^{\beta}}}\leq x^{-\alpha +\frac{a}{\beta}}\int_1^{1-\frac{\alpha}{\beta}}{\frac{\mathrm{d}u}{1+u^{\beta}}}\leq\frac{1}{\left(\beta -1\right)x^{\alpha\left(1-\frac{1}{\beta}\right)}}\rightarrow 0
		\]
	\end{enumerate}
\end{proof}
\end{smybox}
\begin{smybox}[colbacktitle=blue!75!black]{第9题}
	设$f(x)$在$[0,1]$上连续可微且恒不等于$0$,且$\int_0^1{f\left(x\right)}\mathrm{d}x=0$,证明:
	\[\int_0^1{\left| f\left(x\right)\right|}\mathrm{d}x\int_0^1{\left| f'\left(x\right)\right|\mathrm{d}x}>2\int_0^1{f^2\left(x\right)}\mathrm{d}x\]
\begin{proof}
	法1:
	\begin{align*}
	2\int_0^1{f}\left(x\right)^2\mathrm{d}x&=\int_0^1{f}\left(x\right)\left[2f\left(x\right)-f\left(0\right)-f\left(1\right)\right]\mathrm{d}x=\int_0^1{f}\left(x\right)\left[\left(f\left(x\right)-f\left(0\right)\right)-\left(f\left(1\right)-f\left(x\right)\right)\right]\mathrm{d}x\\
	&
	=\int_0^1{f}\left(x\right)\left[\int_0^x{f'}\left(t\right)\mathrm{d}t-\int_x^1{f'}\left(t\right)\mathrm{d}t\right]\mathrm{d}x\leq\int_0^1{\left| f\left(x\right)\right|}\left|\int_0^x{f'}\left(t\right)\mathrm{d}t-\int_x^1{f'}\left(t\right)\mathrm{d}t\right|\mathrm{d}x\\
	&
	\leq\int_0^1{\left| f\left(x\right)\right|}\left(\int_0^x{\left| f'\left(t\right)\right|}\mathrm{d}t+\int_x^1{\left| f'\left(t\right)\right|}\mathrm{d}t\right)\mathrm{d}x=\int_0^1{\left| f\left(x\right)\right|}\left(\int_0^1{\left| f'\left(t\right)\right|}\mathrm{d}t\right)\mathrm{d}x\\
	&=\int_0^1{\left| f'\left(x\right)\right|}\mathrm{d}x\cdot\int_0^1{\left| f\left(x\right)\right|}\mathrm{d}x
	\end{align*}
	法2:令$F\left(x\right)=\int_0^x{f}\left(t\right)\mathrm{d}t$,则有
	\[
	\int_0^1{f^2}\left(x\right)\mathrm{d}x=\left[f\left(x\right)F\left(x\right)\right]\mid_{0}^{1}-\int_0^1{f'}\left(x\right)F\left(x\right)\mathrm{d}x=-\int_0^{1}f'(x)F\left(x\right)\mathrm{d}x
	\]
	因此
	\[
	2\left| F\left(x\right)\right|=\left| F\left(x\right)-F\left(0\right)\right|+\left| F\left(1\right)-F\left(x\right)\right|=\left|\int_0^x{f}\left(t\right)\mathrm{d}t\right|+\left|\int_x^1{f}\left(t\right)\mathrm{d}t\right|\leq\int_0^1{\left| f\left(t\right)\right|}\mathrm{d}t
	\]
%	则有
%	\[\int_0^1{\left| f\left(x\right)\right|}\mathrm{d}x\int_0^1{\left| f'\left(x\right)\right|\mathrm{d}x}>2\int_0^1{f^2\left(x\right)}\mathrm{d}x\]
	%	注意到
	%	\begin{equation}
	%	\int_0^1{f^2}\left(x\right)\mathrm{d}x=\int_0^1{f\left(x\right)}\left(f\left(x\right)-f\left(0\right)\right)\mathrm{d}x=\int_0^1{f\left(x\right)}\left(\int_0^x{f'\left(t\right)}\mathrm{d}t\right)\mathrm{d}x
	%	\end{equation}
	%	\begin{equation}
	%	\int_0^1{f^2}\left(x\right)\mathrm{d}x=\int_0^1{f\left(x\right)\left(f\left(1\right)-f\left(x\right)\right)}\mathrm{d}x=\int_0^1{f\left(x\right)}\left(\int_x^1{f'\left(t\right)}\mathrm{d}t\right)\mathrm{d}x
	%	\end{equation}
	%	两式相加可得
	%	\[
	%	2\int_0^1{f^2}\left(x\right)\mathrm{d}x=\int_0^1{f\left(x\right)\left(\int_0^1{f'\left(x\right)}\mathrm{d}x\right)\mathrm{d}x}<\int_0^1{\left| f\left(x\right)\right|}\mathrm{d}x\int_0^1{\left| f'\left(x\right)\right|\mathrm{d}x}
	%	\]
	
%	法3:对$\int_{0}^{1}\left|f^{\prime}(x)\right| \mathrm{d} x \geq|f(b)-f(a)|, f(b)=\max _{x \in[0,1]} f(x), f(a)=\min _{x \in[0,1]} f(x)$,其中$a \in W=\{f(x)<0\}, b \in V=\{f(x) \geq 0\}$且$\int_{V} f(x) \mathrm{d} x+\int_{W} f(x) \mathrm{d} x=0$,即有
%	\begin{align*}
%	\int_0^1{\left| f'(x)\right|}\mathrm{d}x&		\geq\frac{\int_W{f^2}\left(x\right)\mathrm{d}x}{\int_W{f}\left(x\right)\mathrm{d}x}-\frac{\int_V{f^2}\left(x\right)\mathrm{d}x}{\int_V{f}\left(x\right)\mathrm{d}x}=\frac{\int_W{f^2}\left(x\right)\mathrm{d}x+\int_V{f^2}\left(x\right)\mathrm{d}x}{\int_W{f}\left(x\right)\mathrm{d}x}\\
%	&=\frac{\int_0^1{f^2}\left(x\right)\mathrm{d}x}{\int_W{f}\left(x\right)\mathrm{d}x}=\frac{\int_0^1{f^2}\left(x\right)\mathrm{d}x}{\frac{1}{2}\int_0^1{\left| f\left(x\right)\right|}\mathrm{d}x}
%	\end{align*}
	%		因此$\int_{0}^{1}\left|f^{\prime}(x)\right| \mathrm{d} x \int_{0}^{1}|f(x)| \mathrm{d} x \geq 2 \int_{0}^{1} f^{2}(x) \mathrm{d} x$
\end{proof}
\end{smybox}
\end{tcolorbox}
\end{lstlisting}
\begin{tcolorbox}[colback=yellow!10!white,colframe=red!75!black,title=试题解答]
%	\includegraphics{hzb.pdf}
	  
\newcommand\wqy[1]{\CJKfontspec{文泉驿微米黑}#1}
\newcommand\fzxk[1]{\CJKfontspec{方正行楷简体}#1}
\newcommand\hwcy[1]{\CJKfontspec{华文彩云}#1}
\usetikzlibrary{patterns,decorations.text}
\usetikzlibrary{decorations.text}
\newcommand\fzxkjt[1]{\CJKfontspec{方正行楷简体}#1} 
\setCJKfamilyfont{kd}{华文行楷}

%%%%%%%%%%%%%%%%%%%%%%%%%%%%%%%%%%%%%%%%%%%%%%

\newcommand{\chaoda}{\fontsize{55pt}{\baselineskip}\selectfont}
\newcommand{\chuhao}{\fontsize{42pt}{\baselineskip}\selectfont}     % 字号设置
\newcommand{\xiaochuhao}{\fontsize{36pt}{\baselineskip}\selectfont} % 字号设置
\newcommand{\yihao}{\fontsize{28pt}{\baselineskip}\selectfont}      % 字号设置
\newcommand{\erhao}{\fontsize{21pt}{\baselineskip}\selectfont}      % 字号设置
\newcommand{\xiaoerhao}{\fontsize{18pt}{\baselineskip}\selectfont}  % 字号设置
\newcommand{\sanhao}{\fontsize{15.75pt}{\baselineskip}\selectfont}  % 字号设置
\newcommand{\xiaosanhao}{\fontsize{15pt}{\baselineskip}\selectfont} % 字号设置
\newcommand{\sihao}{\fontsize{14pt}{\baselineskip}\selectfont}      % 字号设置
\newcommand{\xiaosihao}{\fontsize{12pt}{14pt}\selectfont}           % 字号设置
\newcommand{\wuhao}{\fontsize{10.5pt}{12.6pt}\selectfont}           % 字号设置
\newcommand{\xiaowuhao}{\fontsize{9pt}{11pt}{\baselineskip}\selectfont}   % 字号设置
\newcommand{\liuhao}{\fontsize{7.875pt}{\baselineskip}\selectfont}  % 字号设置
\newcommand{\qihao}{\fontsize{5.25pt}{\baselineskip}\selectfont}    % 字号设置
%%%%%%%%%% 定理类环境的定义 %%%%%%%%%%
%% 必须在导入中文环境之后
\newtheorem{example}{Example}             %
\newtheorem{algorithm}{算法}
\newtheorem{theorem}{定理} 
\newtheorem{definition}{定义}
\newtheorem{axiom}{公理}
\newtheorem{property}{性质}
\newtheorem{proposition}{命题}
\newtheorem{lemma}{引理}
\newtheorem{corollary}{推论}
\newtheorem{remark}{注解}
\newtheorem{condition}{条件}
\newtheorem{conclusion}{结论}
\newtheorem{assumption}{假设}
\newtheorem{proof}{证明}
\newtheorem{solution}{解}
\newtcolorbox{smybox}[2][]{enhanced,breakable,
	before skip=2mm,after skip=2mm,
	colback=black!5,colframe=black!50,boxrule=0.2mm,
	attach boxed title to top left={xshift=0.5cm,yshift*=1mm-\tcboxedtitleheight},
	varwidth boxed title*=-0.5cm,
	boxed title style={frame code={
			\path[fill=tcbcol@back!30!black]
			([yshift=-1mm,xshift=-1mm]frame.north west)
			arc[start angle=0,end angle=180,radius=1mm]
			([yshift=-1mm,xshift=1mm]frame.north east)
			arc[start angle=180,end angle=0,radius=1mm];
			\path[left color=tcbcol@back!60!black,right color=tcbcol@back!60!black,
			middle color=tcbcol@back!80!black]
			([xshift=-2mm]frame.north west) -- ([xshift=2mm]frame.north east)
			[rounded corners=1mm]-- ([xshift=1mm,yshift=-1mm]frame.north east)
			-- (frame.south east) -- (frame.south west)
			-- ([xshift=-1mm,yshift=-1mm]frame.north west)
			[sharp corners]-- cycle;
		},interior engine=empty,
	},
fonttitle=\bfseries,
title={#2},#1}

\begin{center}
	{\erhao \CJKfamily{kd}{2019年南开大学数学分析试题解答} }\\
	\vspace{0.2cm}
	{\sihao \textbf{满分:150分,考试时间:150分钟} }\\
	\vspace{0.3cm}
\end{center}

\begin{smybox}[colbacktitle=green!25!black!10!white]{第8题}
	已知$\alpha,\beta$均为正实数,且$\max\left\{\alpha ,\beta\right\}>1$,试证:
	\[
	\lim_{x\rightarrow +\infty}\int_1^x{\frac{1}{x^{\alpha}+t^{\beta}}}\mathrm{d}t=0
	\]
\begin{proof}由题设易知有
	\begin{enumerate} 
		\item 当$\alpha>1$时,有
		\[
		0 \leq \int_{1}^{x} \frac{\mathrm{d}t t}{x^{\alpha}+t^{\beta}} \leq \int_{1}^{x} \frac{1}{x^{\alpha}} \mathrm{d}tt=\frac{x-1}{x^{\alpha}} \rightarrow 0(x \rightarrow \infty)
		\]
		\item 当$\beta>1,0<\alpha \leq 1$时,且$\left(x\rightarrow\infty\right)$有
		\[
		\int_1^x{\frac{\mathrm{d}t}{x^{\alpha}+t^{\beta}}}=x^{-\alpha +\frac{\alpha}{\beta}}\int_1^x{\frac{\mathrm{d}\left(x^{-\frac{\alpha}{\beta}}t\right)}{1+\left(x^{-\frac{\alpha}{\beta}}t\right)^{\beta}}}\leq x^{-\alpha +\frac{a}{\beta}}\int_1^{1-\frac{\alpha}{\beta}}{\frac{\mathrm{d}u}{1+u^{\beta}}}\leq\frac{1}{\left(\beta -1\right)x^{\alpha\left(1-\frac{1}{\beta}\right)}}\rightarrow 0
		\]
	\end{enumerate}
\end{proof}
\end{smybox}
\begin{smybox}[colbacktitle=blue!75!black]{第9题}
	设$f(x)$在$[0,1]$上连续可微且恒不等于$0$,且$\int_0^1{f\left(x\right)}\mathrm{d}x=0$,证明:
	\[\int_0^1{\left| f\left(x\right)\right|}\mathrm{d}x\int_0^1{\left| f'\left(x\right)\right|\mathrm{d}x}>2\int_0^1{f^2\left(x\right)}\mathrm{d}x\]
\begin{proof}
	法1:
	\begin{align*}
	2\int_0^1{f}\left(x\right)^2\mathrm{d}x&=\int_0^1{f}\left(x\right)\left[2f\left(x\right)-f\left(0\right)-f\left(1\right)\right]\mathrm{d}x=\int_0^1{f}\left(x\right)\left[\left(f\left(x\right)-f\left(0\right)\right)-\left(f\left(1\right)-f\left(x\right)\right)\right]\mathrm{d}x\\
	&
	=\int_0^1{f}\left(x\right)\left[\int_0^x{f'}\left(t\right)\mathrm{d}t-\int_x^1{f'}\left(t\right)\mathrm{d}t\right]\mathrm{d}x\leq\int_0^1{\left| f\left(x\right)\right|}\left|\int_0^x{f'}\left(t\right)\mathrm{d}t-\int_x^1{f'}\left(t\right)\mathrm{d}t\right|\mathrm{d}x\\
	&
	\leq\int_0^1{\left| f\left(x\right)\right|}\left(\int_0^x{\left| f'\left(t\right)\right|}\mathrm{d}t+\int_x^1{\left| f'\left(t\right)\right|}\mathrm{d}t\right)\mathrm{d}x=\int_0^1{\left| f\left(x\right)\right|}\left(\int_0^1{\left| f'\left(t\right)\right|}\mathrm{d}t\right)\mathrm{d}x\\
	&=\int_0^1{\left| f'\left(x\right)\right|}\mathrm{d}x\cdot\int_0^1{\left| f\left(x\right)\right|}\mathrm{d}x
	\end{align*}
	法2:令$F\left(x\right)=\int_0^x{f}\left(t\right)\mathrm{d}t$,则有
	\[
	\int_0^1{f^2}\left(x\right)\mathrm{d}x=\left[f\left(x\right)F\left(x\right)\right]\mid_{0}^{1}-\int_0^1{f'}\left(x\right)F\left(x\right)\mathrm{d}x=-\int_0^{1}f'(x)F\left(x\right)\mathrm{d}x
	\]
	因此
	\[
	2\left| F\left(x\right)\right|=\left| F\left(x\right)-F\left(0\right)\right|+\left| F\left(1\right)-F\left(x\right)\right|=\left|\int_0^x{f}\left(t\right)\mathrm{d}t\right|+\left|\int_x^1{f}\left(t\right)\mathrm{d}t\right|\leq\int_0^1{\left| f\left(t\right)\right|}\mathrm{d}t
	\]
%	则有
%	\[\int_0^1{\left| f\left(x\right)\right|}\mathrm{d}x\int_0^1{\left| f'\left(x\right)\right|\mathrm{d}x}>2\int_0^1{f^2\left(x\right)}\mathrm{d}x\]
	%	注意到
	%	\begin{equation}
	%	\int_0^1{f^2}\left(x\right)\mathrm{d}x=\int_0^1{f\left(x\right)}\left(f\left(x\right)-f\left(0\right)\right)\mathrm{d}x=\int_0^1{f\left(x\right)}\left(\int_0^x{f'\left(t\right)}\mathrm{d}t\right)\mathrm{d}x
	%	\end{equation}
	%	\begin{equation}
	%	\int_0^1{f^2}\left(x\right)\mathrm{d}x=\int_0^1{f\left(x\right)\left(f\left(1\right)-f\left(x\right)\right)}\mathrm{d}x=\int_0^1{f\left(x\right)}\left(\int_x^1{f'\left(t\right)}\mathrm{d}t\right)\mathrm{d}x
	%	\end{equation}
	%	两式相加可得
	%	\[
	%	2\int_0^1{f^2}\left(x\right)\mathrm{d}x=\int_0^1{f\left(x\right)\left(\int_0^1{f'\left(x\right)}\mathrm{d}x\right)\mathrm{d}x}<\int_0^1{\left| f\left(x\right)\right|}\mathrm{d}x\int_0^1{\left| f'\left(x\right)\right|\mathrm{d}x}
	%	\]
	
%	法3:对$\int_{0}^{1}\left|f^{\prime}(x)\right| \mathrm{d} x \geq|f(b)-f(a)|, f(b)=\max _{x \in[0,1]} f(x), f(a)=\min _{x \in[0,1]} f(x)$,其中$a \in W=\{f(x)<0\}, b \in V=\{f(x) \geq 0\}$且$\int_{V} f(x) \mathrm{d} x+\int_{W} f(x) \mathrm{d} x=0$,即有
%	\begin{align*}
%	\int_0^1{\left| f'(x)\right|}\mathrm{d}x&		\geq\frac{\int_W{f^2}\left(x\right)\mathrm{d}x}{\int_W{f}\left(x\right)\mathrm{d}x}-\frac{\int_V{f^2}\left(x\right)\mathrm{d}x}{\int_V{f}\left(x\right)\mathrm{d}x}=\frac{\int_W{f^2}\left(x\right)\mathrm{d}x+\int_V{f^2}\left(x\right)\mathrm{d}x}{\int_W{f}\left(x\right)\mathrm{d}x}\\
%	&=\frac{\int_0^1{f^2}\left(x\right)\mathrm{d}x}{\int_W{f}\left(x\right)\mathrm{d}x}=\frac{\int_0^1{f^2}\left(x\right)\mathrm{d}x}{\frac{1}{2}\int_0^1{\left| f\left(x\right)\right|}\mathrm{d}x}
%	\end{align*}
	%		因此$\int_{0}^{1}\left|f^{\prime}(x)\right| \mathrm{d} x \int_{0}^{1}|f(x)| \mathrm{d} x \geq 2 \int_{0}^{1} f^{2}(x) \mathrm{d} x$
\end{proof}
\end{smybox}
\end{tcolorbox}

\subsection{改变边框}
\begin{lstlisting}[language={TeX}]
\begin{tcolorbox}[enhanced,sharp corners=uphill,
colback=blue!50!white,colframe=blue!25!black,coltext=yellow,
fontupper=\Large\bfseries,arc=6mm,boxrule=2mm,boxsep=5mm,
borderline={0.3mm}{0.3mm}{white}]
Funny settings.
\end{tcolorbox}
\end{lstlisting}
\begin{tcolorbox}[enhanced,sharp corners=uphill,
	colback=blue!50!white,colframe=blue!25!black,coltext=yellow,
	fontupper=\Large\bfseries,arc=6mm,boxrule=2mm,boxsep=5mm,
	borderline={0.3mm}{0.3mm}{white}]
	Funny settings.
\end{tcolorbox}

\subsection{给盒子添加背景}
\begin{lstlisting}[language={TeX}]
\begin{tcolorbox}[enhanced,frame style image=blueshade.png,
opacityback=0.75,opacitybacktitle=0.25,
colback=blue!5!white,colframe=blue!75!black,
title=My title]
This box is filled with an external image.\par
Title and interior are made partly transparent to show the image.
\end{tcolorbox}
\end{lstlisting}
\begin{tcolorbox}[enhanced,frame style image=blueshade.png,
	opacityback=0.75,opacitybacktitle=0.25,
	colback=blue!5!white,colframe=blue!75!black,
	title=My title]
	This box is filled with an external image.\par
	Title and interior are made partly transparent to show the image.
\end{tcolorbox}

\subsection{重定义新的box设置标题居中}
\begin{lstlisting}[language={TeX}]
\begin{tcolorbox}[enhanced,attach boxed title to top center={yshift=-3mm,yshifttext=-1mm},
colback=blue!5!white,colframe=blue!75!black,colbacktitle=red!80!black,
title=`给框加标题`,fonttitle=\bfseries,
boxed title style={size=small,colframe=red!50!black} ]
This box uses a \textit{boxed title}. The box of the title can
be formatted independently from the main box.
\end{tcolorbox}
\end{lstlisting}
\begin{tcolorbox}[enhanced,attach boxed title to top center={yshift=-3mm,yshifttext=-1mm},
	colback=blue!5!white,colframe=blue!75!black,colbacktitle=red!80!black,
	title=给框加标题,fonttitle=\bfseries,
	boxed title style={size=small,colframe=red!50!black} ]
	This box uses a \textit{boxed title}. The box of the title can
	be formatted independently from the main box.
\end{tcolorbox}
\begin{lstlisting}[language={TeX}]
\newtcolorbox{Mybox}[2][]
{colback = red!5!white, colframe = red!75!black, fonttitle = \bfseries,
colbacktitle = red!85!black, enhanced,
attach boxed title to top center={yshift=-2mm},
title=#2,#1}
\begin{mybox}[colback=yellow]{Hello there}
This is my own box with a mandatory title
and options.
\end{mybox}
\end{lstlisting}
\begin{Mybox}[colback=yellow]{Hello there}
	This is my own box with a mandatory title
	and options.
\end{Mybox}
%----------------------------------------------------------
\section{LaTeX代码和显示效果并列展示}
如果把可选参数里的标题相关的选项``title,fonttitle''去掉,会产生没有表题栏的框。
\subsection{平行}
\begin{lstlisting}[language={TeX}]
\begin{tcblisting}{colback=red!5!white,colframe=red!75!black,title=`平行展示`,fonttitle=\bfseries}
This is a \LaTeX\ example:
\begin{equation}
\sum\limits_{i=1}^n i = \frac{n(n+1)}{2}.
\end{equation}
\end{tcblisting}
\end{lstlisting}
%\begin{tcblisting}{colback=red!5!white,colframe=red!75!black,title=平行展示,fonttitle=\bfseries}
%	This is a \LaTeX\ example:
%	\begin{equation}
%	\sum\limits_{i=1}^n i = \frac{n(n+1)}{2}.
%	\end{equation}
%\end{tcblisting}

\subsection{并列}
\begin{lstlisting}[language={TeX}]
\begin{tcblisting}{colback=red!5!white,colframe=red!75!black,listing side text,title=`并列展示`,fonttitle=\bfseries}
This is a \LaTeX\ example:
\begin{equation}
\sum\limits_{i=1}^n i = \frac{n(n+1)}{2}.
\end{equation}
\end{tcblisting}
\end{lstlisting}
%\begin{tcblisting}{colback=red!5!white,colframe=red!75!black,listing side text,
%		title=并列展示,fonttitle=\bfseries}
%	This is a \LaTeX\ example:
%	\begin{equation}
%	\sum\limits_{i=1}^n i = \frac{n(n+1)}{2}.
%	\end{equation}
%\end{tcblisting}
%----------------------------------------------------------
\section{Theorems 程序包}
tcolorbox 宏包提供了 theorems 程序包来实现定理类的环境。theorems 程序包会自动加载 amsmath 宏包。加载调用的方法如摘要所言.
\begin{lstlisting}[language={TeX}]
\newtcbtheorem[⟨init options⟩]{⟨name⟩}{⟨display name⟩}{⟨options⟩}{⟨prefix⟩}
\renewtcbtheorem[⟨init options⟩]{⟨name⟩}{⟨display name⟩}{⟨options⟩}{⟨prefix⟩}
\end{lstlisting}

两个命令分别都有 4 个必需参数和 1 个可选参数。name: 创建的 LaTeX 环境名称;display name: 创建的环境的标题名称;options: 传入 tcolorbox 的参数;prefix: 用于生成环境的 label;init options: 用于控制编号.
\begin{lstlisting}[language={TeX}]
\newtcbtheorem[number within=section]{mytheo}{定理}%
{colback=green!5,colframe=green!35!black,fonttitle=\bfseries}{th}
\begin{mytheo}{我的定理}{}
这是默认样式。
\end{mytheo}
\begin{mytheo}[separator sign = {\ $\blacktriangleright$}]{我的定理}{}
分隔符修改为 $\blacktriangleright$。
\end{mytheo}
\begin{mytheo}[description delimiters parenthesis]{我的定理}{}
定界符修改为圆括号。
\end{mytheo}
\begin{mytheo}[description color=red!25!yellow,
description font= {\mdseries\itshape}]{我的定理}{}
标题的字体及颜色修改。
\end{mytheo}
\begin{mytheo}[terminator sign={.}]{我的定理}{}
标题后的终止符。
\end{mytheo}
\end{lstlisting}
\newtcbtheorem[number within=section]{mytheo}{定理}%
{colback=green!5,colframe=green!35!black,fonttitle=\bfseries}{th}
\begin{mytheo}{我的定理}{}
	这是默认样式。
\end{mytheo}
\begin{mytheo}[separator sign = {\ $\blacktriangleright$}]{我的定理}{}
	分隔符修改为 $\blacktriangleright$。
\end{mytheo}
\begin{mytheo}[description delimiters parenthesis]{我的定理}{}
	定界符修改为圆括号。
\end{mytheo}
\begin{mytheo}[description color=red!25!yellow,
	description font= {\mdseries\itshape}]{我的定理}{}
	标题的字体及颜色修改。
\end{mytheo}
\begin{mytheo}[terminator sign={.}]{我的定理}{}
	标题后的终止符。
\end{mytheo}
\begin{lstlisting}[language={TeX}]
\newtcbtheorem[auto counter,number within=section]{theo}%
{Theorem}{fonttitle=\bfseries\upshape, fontupper=\slshape,
arc=0mm, colback=blue!5!white,colframe=blue!75!black}{theorem}
\end{lstlisting}
\newtcbtheorem[auto counter,number within=section]{theo}%
{Theorem}{fonttitle=\bfseries\upshape, fontupper=\slshape,
	arc=0mm, colback=blue!5!white,colframe=blue!75!black}{theorem}
\begin{theo}{Summation of Numbers}{summation}
	For all natural number $n$ it holds:
	\begin{equation}
	\tcbhighmath{\sum\limits_{i=1}^n i = \frac{n(n+1)}{2}.}
	\end{equation}
\end{theo}

We have given Theorem \ref{theorem:summation} on page \pageref{theorem:summation}.

\newtcbtheorem[use counter from=theo]{antheo}%
{Theorem}{theorem style=change,oversize,enlarge top by=1mm,enlarge bottom by=1mm,
	enhanced jigsaw,interior hidden,fuzzy halo=1mm with green,
	fonttitle=\bfseries\upshape,fontupper=\slshape,
	colframe=green!75!black,coltitle=green!50!blue!75!black}{antheorem}

\begin{antheo}{Summation of Numbers}{summation}
	For all natural number $n$ it holds:
	\begin{equation}
	\tcbhighmath{\sum\limits_{i=1}^n i = \frac{n(n+1)}{2}.}
	\end{equation}
\end{antheo}
%----------------------------------------------------------
\section{加图片水印}
\begin{lstlisting}[language={TeX}]
\begin{tcolorbox}[enhanced,watermark graphics=1.jpg,
watermark opacity=0.3,watermark zoom=0.9,
colback=green!5!white,colframe=green!75!black,
fonttitle=\bfseries, title=盒子中添加图片水印]
数学分析作为数学专业考研的专业课之一,也是考查学生的基本功,在复试中会出现
实变函数与近世代数与英语口语的考核,不同高校对学生的要求是不一样的。针对过去了
三个月的 2019 年考研,我总结部分真题的解析,对不同院校的考查有了整理把控。
\end{tcolorbox}
\end{lstlisting}
\begin{tcolorbox}[enhanced,watermark graphics=1.jpg,
	watermark opacity=0.3,watermark zoom=0.9,
	colback=green!5!white,colframe=green!75!black,
	fonttitle=\bfseries, title=盒子中添加图片水印]
	数学分析作为数学专业考研的专业课之一,也是考查学生的基本功,在复试中会出现
	实变函数与近世代数与英语口语的考核,不同高校对学生的要求是不一样的。针对过去了
	三个月的 2019 年考研,我总结部分真题的解析,对不同院校的考查有了整理把控。
\end{tcolorbox}

%----------------------------------------------------------
\section{盒子中嵌入盒子}
\begin{tcolorbox}[colback=yellow!10!white,colframe=yellow!50!black,
	every box/.style={fonttitle=\bfseries},title=Box]
	\begin{tcolorbox}[enhanced,colback=red!10!white,colframe=red!50!black,
		colbacktitle=red!85!black,
		title=Box inside box,drop fuzzy shadow]
		\begin{tcolorbox}[beamer,colframe=blue!50!black,title=Box inside box inside box]
			And now for something completely different: Boxes!\par\medskip
			\newtcbox{\mybox}[1][]{nobeforeafter,tcbox raise base,colframe=green!50!black,colback=green!10!white,
				sharp corners,top=1pt,bottom=1pt,before upper=\strut,#1}
			\mybox[rounded corners=west]{This} \mybox{is} \mybox{another} \mybox[rounded corners=east]{box.}
		\end{tcolorbox}
	\end{tcolorbox}	
\end{tcolorbox}
%----------------------------------------------------------
\section{适合的盒子}
\begin{lstlisting}[language={TeX}]
\begin{tcolorbox}[enhanced,fit to height=6cm,
colback=green!25!black!10!white,colframe=green!75!black,title=Fit box (10cm),
drop fuzzy shadow,watermark color=white,watermark text=八一考研数学竞赛]

数学分析作为数学专业考研的专业课之一,也是考查学生的基本功,在复试中会出现
实变函数与近世代数与英语口语的考核,不同高校对学生的要求是不一样的。针对过去了
三个月的 2019 年考研,我总结部分真题的解析,对不同院校的考查有了整理把控。
\end{tcolorbox}
\end{lstlisting}
``fit to height''制定框的高度,框里的内容会自动调整到合适的字体大小以适应该高度。
\begin{tcolorbox}[enhanced,fit to height=6cm,
	colback=green!25!black!10!white,colframe=green!75!black,title=Fit box (10cm),
	drop fuzzy shadow,watermark color=white,watermark text=八一考研数学竞赛]
	
	数学分析作为数学专业考研的专业课之一,也是考查学生的基本功,在复试中会出现
	实变函数与近世代数与英语口语的考核,不同高校对学生的要求是不一样的。针对过去了
	三个月的 2019 年考研,我总结部分真题的解析,对不同院校的考查有了整理把控。
\end{tcolorbox}

\begin{tcolorbox}[enhanced,fit to height=2cm,
	colback=green!25!black!10!white,colframe=green!75!black,title=Fit box (5cm),
	drop fuzzy shadow,watermark color=white,watermark text=八一考研数学竞赛]
	数学分析作为数学专业考研的专业课之一,也是考查学生的基本功,在复试中会出现
	实变函数与近世代数与英语口语的考核,不同高校对学生的要求是不一样的。针对过去了
	三个月的 2019 年考研,我总结部分真题的解析,对不同院校的考查有了整理把控。
\end{tcolorbox}
%----------------------------------------------------------
\section{可断的框}
该框的内容如果在当前页不能完全显示,可以自动顺延到下一页,产生断裂的框。
\begin{tcolorbox}[enhanced jigsaw,breakable,pad at break*=1mm,
	colback=blue!5!white,colframe=blue!75!black,title=Breakable box,
	watermark color=white,watermark text=\Roman{tcbbreakpart}]
	数学分析作为数学专业考研的专业课之一,也是考查学生的基本功,在复试中会出现
	实变函数与近世代数与英语口语的考核,不同高校对学生的要求是不一样的。针对过去了
	三个月的 2019 年考研,我总结部分真题的解析,对不同院校的考查有了整理把控。
\end{tcolorbox}
\section{致谢}
本文档是我对 Thomas F. Sturm 的《The tcolorbox Packages-Manual for version 4.15》(\url{https://www.complang.tuwien.ac.at/doc/texlive-latex-extra-doc/latex/tcolorbox/tcolorbox.pdf}) 中的例子进行的实践与笔记,同时参考了龙叔的博文一文《 tcolorbox 宏包简明教程》(\url{https://liam.page/2016/07/22/using-the-tcolorbox-package-to-create-a-new-theorem-environment/})

Happy tcb-ing!
\end{document}