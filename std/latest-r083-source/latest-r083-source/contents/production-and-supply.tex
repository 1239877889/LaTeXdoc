%# -*- coding: utf-8 -*-
\chapter{生产与供给}
\label{sec:production-and-supply}

\section{生产函数}

\begin{Definition}[生产函数]\index{product function 生产函数}
表示在一定时期内,在技术水平不变的情况下,生产中所使用的各种生产要素的数量与所能生产的最大产量之间的关系。
\end{Definition}

用$X_i$表示某产品生产所需之要素投入数量,$Q$表示所能生产的最大产量,则生产函数可以表示成:
\[
Q=f(X_1, X_2, \cdots , X_n)
\]
为简化分析,我们只研究劳动和资本这两种最主要的生产要素所构成的生产函数:
\[
Q=f(L,K)
\]

\section{常见的生产函数}
\subsection{固定替代比例生产函数}
\subsection{固定投入比例生产函数}
\subsection{柯布—道格拉斯生产函数}

\section{短期生产:一种可变要素}
\subsection{生产的长期和短期}
\subsection{总产量、平均产量、边际产量}
在生产函数$Q=f(L,K)$的基础上,假定资本投入量在短期内是固定的,而劳动投入量是可变的,则生产函数可以写成:
\begin{equation}
Q = f(L,\overline K )
\end{equation}

在此基础上很容易得到劳动要素不断增加过程中的总产量($TP_L$)、平均产量($AP_L$)以及边际产量($MP_L$):
\begin{align}
TP_L	&= f(L,\overline K)\\
AP_L	&= \frac{f(L,\overline K)}{L}\\
MP_L	&= \frac{{df(L,\overline K)}}{{dL}}
\end{align}

\subsection{边际报酬}


\section{长期生产:所有要素可变}

\subsection{边际技术替代率}

\section{等产量线}

\section{规模报酬}

\section{最优的生产要素组合}










\section*{推荐阅读}
\markright{推荐阅读}
\addcontentsline{toc}{section}{\hspace{-2.5em}推荐阅读}