\chapter{消费者最优选择}
\label{sec:utility-maximization-and-choice}

消费者最优选择就像在后院刨坑种树,面临两个约束:(1)地点必须在你家的篱笆墙内;(2)必须是生长环境最好的点。当然如果不会引起争端的话,你也可以砍掉篱笆把树种在这个邻里边界上,这就是角点解。总而言之,消费者最优选择必须是\emph{最富偏好}、\emph{负担得起}的消费束。这和笔记开头对需求定义中的\emph{愿意}且\emph{能够}购买很相似,因为他们本来就是一根绳子上的铃铛,扯一扯都会响叮当。

假设某人对两商品的偏好表现为效用函数$u(x,y) = x_1^{\alpha} x_2^{\beta}$,商品价格分别为$p_1$、$p_2$,其收入水平为$m$,其中$\alpha, \beta,x_1,x_2,p_1,p_2,m > 0$。那么对于任一给定的$(p_1,p_2,m)$组合该消费者该如何选择这两种商品的消费数量?

\section{直接的边际分析法}
沿着预算线调整分配额度,使得达到某一个分配状态,任何增减都不会带来额外的改善。
\begin{equation}
MRS_{1,~2}=\frac{p_1}{p_2}
\label{eq:zuiyoufenxi-bianjifenxi-mrs}
\end{equation}

\begin{equation}
\frac{MU_1}{p_1}=\frac{MU_2}{p_2}
\label{eq:zuiyoufenxi-bianjifenxi-a}
\end{equation}

\section{消元法}
假如效用函数为二元函数:
\begin{equation}
u=f(x_1,x_2)
\label{eq:zuiyouxiaofei-eryuanxiaoyong}
\end{equation}
预算约束为$p_1 x_1 + p_2 x_2 = m$,用$x_1$表示$x_2$得到:
\begin{equation}
x_2=\frac{m-p_1 x_1}{p_2}
\label{eq:zuiyouxiaofei-yusuanyueshu-0}
\end{equation}

将式(\ref{eq:zuiyouxiaofei-yusuanyueshu-0})带入式(\ref{eq:zuiyouxiaofei-eryuanxiaoyong})消元为:
\begin{equation}
u=g(x_1)
\label{eq:zuiyouxiaofei-yiyuanxiaoyong}
\end{equation}
该效用函数达到极大值的一阶必要条件为:
\begin{equation}
MU_1 = \frac{du}{dx_1} = g'(x_1) = 0
\label{eq:zuiyouxiaofei-yiyuanxiaoyong-yijiebiyao}
\end{equation}

注意,在这里$MU_1=0$和上一节边际分析中消费者均衡条件(\ref{eq:zuiyoufenxi-bianjifenxi-a})是一致的。因为对于一元效用函数(\ref{eq:zuiyouxiaofei-yiyuanxiaoyong})来说$MU_2=\frac{\partial g(x_1)}{\partial x_2}=0$,于是仍满足式(\ref{eq:zuiyoufenxi-bianjifenxi-a})。



\section{拉格朗日极值法}

首先,其消费束所带来的效用函数值越大越好:
\begin{equation}
\max~u = x_1^{\alpha} x_2^{\beta}
\label{eq:zuiyouxiaofei-xiaoyonghanshu}
\end{equation}

第二,他所选择的消费束必须是可支付的,即面临预算约束:
\begin{equation}
p_1 x_1 + p_2 x_2 \le m
\label{eq:zuiyouxiaofei-yusuanyueshu}
\end{equation}

构造\emph{拉格朗日函数}:
\begin{equation}
\mathcal{L}(x_1,x_2,\lambda)=x_1^{\alpha} x_2^{\beta}+\lambda(m-p_1 x_1 - p_2 x_2)
\label{eq:zuiyouxiaofei-lagelangri}
\end{equation}
一阶必要条件满足
\begin{align}
\frac{\partial \mathcal{L}}{\partial x_1} &= \alpha x_1^{\alpha-1}x_2^{\beta} - \lambda p_1 = 0\\
\frac{\partial \mathcal{L}}{\partial x_2} &= \beta x_1^{\alpha} x_2^{\beta-1} - \lambda p_2 = 0\\
\frac{\partial \mathcal{L}}{\partial \lambda} &= m - p_1 x_1 - p_2 x_2 = 0
\end{align}
消去$\lambda$解得该消费者对这两种商品的最优选择如下,
\begin{align}
x_1 &= \frac{\alpha}{\alpha+\beta} \cdot \frac{m}{p_1} \label{eq:c-d-utility-optimal-choice-x1}\\
x_2 &= \frac{\beta}{\alpha+\beta} \cdot \frac{m}{p_2}
\end{align}

我们分析一下\emph{柯布—道格拉斯函数}最优解可爱的数学形式,以式(\ref{eq:c-d-utility-optimal-choice-x1})为例:
\[
x_1 = \frac{\alpha}{\alpha+\beta} \cdot \frac{m}{p_1} = \dfrac{m \cdot \dfrac{\alpha}{\alpha+\beta}}{p_1}
\]
很明显,其中$m \cdot \frac{\alpha}{\alpha+\beta}$是收入中用来购买$x_1$的部分,$\frac{\alpha}{\alpha+\beta}$即是商品1在收入中所占的\emph{消费份额}。如果我们增设或者说对现有的效用函数做一个单调变换,对其开$\alpha+\beta$次方得到:
\begin{equation}
u(x,y) = x_1^{c} x_2^{d}
\label{eq:zuiyouxuanze-cd-utility-fenexingshi}
\end{equation}
其中$c=\frac{\alpha}{\alpha+\beta}$,$d=\frac{\beta}{\alpha+\beta}$,$c+d=1$。则$c$、$d$直接地代表两商品在收入中所占的消费份额。在这种新的表达形式下消费者最优选择变为:
\begin{align}
x_1 &= c \cdot \frac{m}{p_1}\\
x_2 &= d \cdot \frac{m}{p_2}
\end{align}

可见柯布—道格拉斯效用函数中两商品的需求曲线为双曲线的一支,其需求的价格弹性、需求的交叉价格弹性以及需求的收入弹性为:
\begin{align}
\varepsilon_{1,~1} &= \frac{dx_1}{dp_1}\frac{p_1}{x_1} = -1\\
\varepsilon_{2,~1} &= \frac{dx_1}{dp_2}\frac{p_2}{x_1} = 0\\
\varepsilon_{m}    &= \frac{dx_1}{dm}\frac{m}{x_1} = 1
\end{align}

\section{拟线性偏好的非线性规划\footnote{%
张军. 高级微观经济学 [M]. 上海: 复旦大学出版社, 2002: 9--11.}}
\begin{equation}
\begin{split}
        \max~u &= x_2 + a \ln x_1\\
\text{s.t.~} m &\ge p_1 x_1 + p_2 x_2\\
\text{and~} x_1&{,~}x_2 \ge 0
\end{split}
\end{equation}

构造\emph{拉格朗日函数}$\mathcal{L}(x_1, x_2, \lambda)=x_2 + a \ln x_1 + \lambda (m - p_1 x_1 -p_2 x_2)$,一阶必要条件(\emph{库恩—塔克条件}):
\begin{equation}
\begin{split}
\frac{\partial \mathcal{L}}{\partial x_1} &= \frac{a}{x_1} - \lambda p_1 \le 0, \quad x_1 \ge 0,~x_1 \frac{\partial \mathcal{L}}{\partial x_1} = 0\\
\frac{\partial \mathcal{L}}{\partial x_2} &= m - \lambda p_2 \le 0, \quad x_2 \ge 0,~x_2 \frac{\partial \mathcal{L}}{\partial x_2} = 0\\
\frac{\partial \mathcal{L}}{\partial \lambda} &= m - p_1 x_1 - p_2 x_2 \ge 0, \quad \lambda \ge 0,~\lambda \frac{\partial \mathcal{L}}{\partial \lambda} = 0
\end{split}
\end{equation}

一般情况下,要讨论$x_1 > 0$,$x_1=0$;$x_2 > 0$,$x_2 = 0$与$\lambda > 0$,$\lambda = 0$组合形成的8种情况,不过在此问题中,边际效用$u_1$,$u_2$均为正(即偏好关系满足局部非饱和性),故而不会有收入剩余,否则可以通过继续增加消费使得效用增加,因此预算约束是紧的,即$\lambda > 0$;另外,由效用函数的形式,必有$x_1 > 0$。因此需要讨论的情况只有两种:

第一种情况:$x_1 > 0$,$x_2 = 0$,$\lambda > 0$。此时\emph{库恩—塔克条件}变为:
\begin{equation}
\begin{split}
\frac{\partial \mathcal{L}}{\partial x_1} &= \frac{a}{x_1} - \lambda p_1 = 0\\
\frac{\partial \mathcal{L}}{\partial x_2} &= m - \lambda p_2 \le 0\\
\frac{\partial \mathcal{L}}{\partial \lambda} &= m - p_1 x_1 - p_2 x_2 = 0
\end{split}
\end{equation}
解得:$x_1 = \frac{m}{p_1}$,$x_2 = 0$,$\lambda = \frac{a}{m}$,并且满足参数条件$m \le ap_2$。

第二种情况:$x_1 > 0$,$x_2 > 0$,$\lambda > 0$。此时\emph{库恩—塔克条件}变为:
\begin{equation}
\begin{split}
\frac{\partial \mathcal{L}}{\partial x_1} &= \frac{a}{x_1} - \lambda p_1 = 0\\
\frac{\partial \mathcal{L}}{\partial x_2} &= m - \lambda p_2 = 0\\
\frac{\partial \mathcal{L}}{\partial \lambda} &= m - p_1 x_1 - p_2 x_2 = 0
\end{split}
\end{equation}
解得:$x_1 = \frac{ap_2}{p_1}$,$x_2 = \frac{m - ap_2}{p_2}$,$\lambda = \frac{1}{p_2}$,并且满足参数条件$m > ap_2$。

从求解的过程来看,虽然题目没有直接说明,但其实问题中的参数是有条件的,参数空间可以划分为各个不同的部分,每一部分对应于求解的一个具体过程。当$m \le ap_2$时,$x_1 > 0$,$x_2 > 0$,$\lambda > 0$;当$m > ap_2$时,$x_1 > 0$,$x_2 > 0$,$\lambda > 0$。

\section*{推荐阅读}
\markright{推荐阅读}
\addcontentsline{toc}{section}{\hspace{-2.5em}推荐阅读}

\newpage
\section*{本章附录}
\markright{本章附录}
\addcontentsline{toc}{section}{\hspace{-2.5em}本章附录}
\label{sec:appendix-utility-maximization-and-choice}
