%# -*- coding:utf-8 -*-
\usepackage{ifxetex}
\usepackage[tbtags]{amsmath}
\usepackage{amssymb}
\usepackage{mathrsfs}
\usepackage{fontspec}
\usepackage{calc}
\usepackage[usenames,svgnames,dvipsnames]{xcolor}
\usepackage{graphicx}
\usepackage{graphics}
\usepackage{graphfig}
\usepackage[numbers,sort&compress,super,square]{natbib} %参考文献[1-4]
\usepackage{fancybox}
\usepackage{fancynum}	%12345->12,345
\usepackage{framed}		%定理边框需要
\usepackage{listings}
\usepackage{makeidx}
\usepackage{rotating}
\usepackage{tabularx}	%列宽自适应表格
\usepackage{multirow}
\usepackage{pst-barcode}
\usepackage{auto-pst-pdf}
\usepackage[
	bookmarks=false,
	linktocpage,
	unicode=true,
	CJKbookmarks=true,
	citecolor=red,
	colorlinks=true,
	linkcolor=blue]{hyperref}
%PDF meta 信息
\hypersetup{
	pdfinfo={
	Title={ecoNote: Microeconomics},
	Author={Nort Silent},
	Subject={Microeconomics},
	Keywords={微观经济学, 笔记, 高鸿业, 范里安, 尼克尔森}
	}
}

%PDF文件书签设置,以bookmark渠道hyperref生成书签
\usepackage{bookmark}
\bookmarksetup{open,numbered,openlevel=0}

%列表设置
\usepackage{paralist}
\setdefaultleftmargin{2em}{}{}{}{.5em}{.5em}

%-------浮动体设置----------------------------------------------------
\usepackage{flafter}                %确保浮动体在页面中不出现在其代码插入点之前
\usepackage[section]{placeins}      %确保浮动体不出现在section之外

%-------纸张----------------------------------------------------------
\usepackage[paperwidth=176mm,
			paperheight=250mm,
            text={140mm,210mm},
            inner=15mm,
			outer=25mm,
			marginpar=22mm,
			driver=xetex,
            includehead,
            vmarginratio=1:1]{geometry}
\setlength{\headheight}{16pt}
\setlength{\headsep}{16pt}
\renewcommand{\headrulewidth}{0pt}  %页眉线宽,0为无页眉线
%页面基本布局样式
\pagestyle{fancy}
\fancyhf{}
\fancyhead[EL,OR]{\rmfamily{\thepage}}
\fancyhead[OL]{\rightmark}
\fancyhead[ER]{\leftmark}

\fancypagestyle{plain}{
\fancyhf{}
\renewcommand{\headrulewidth}{0pt}
\renewcommand{\headrule}{}
}
%空白页
\makeatletter
\def\cleardoublepage{\clearpage\if@twoside \ifodd\c@page\else
\hbox{}
\vspace*{\fill}
\begin{center}
\emph{空}
\end{center}
\vspace{\fill}
\thispagestyle{empty}
\newpage
\if@twocolumn\hbox{}\newpage\fi\fi\fi}
\makeatother
%文档中临时变更页面设置
%\usepackage{changepage}

%------字体设置-------------------------------------------------------------
\usepackage[loosequotes,lf,onlymath]{MinionPro} %https://code.google.com/p/pixtex/ (pixtex-adobe-MinionPro-MnSymbol-fonts.7z)
\usepackage{wasysym}

%西文字体
\setmainfont{Minion Pro}
\setsansfont[
          Extension=.ttf ,
           BoldFont={Helvetica Bold},
         ItalicFont={Helvetica Oblique},
     BoldItalicFont={Helvetica Bold Oblique},
        SlantedFont={Helvetica Oblique},
    BoldSlantedFont={Helvetica Bold Oblique}]{Helvetica}
\setmonofont[Scale=MatchLowercase]{Monaco}

%中文字体
\setCJKmainfont[
           BoldFont={Adobe Heiti Std},
         ItalicFont=DFKaiW5-A]{FZShuSong-Z01}
\setCJKsansfont{Adobe Heiti Std}
\setCJKmonofont{Adobe Fangsong Std}

%致辞页面用字(手写体、书法字)
%西文
\newfontfamily\zapf[Style=Swash]{Zapfino Extra LT Pro}
%中文
\setCJKfamilyfont{fzsigns}{FZYingBiXingShu-S16S} % 方正硬笔行书
\newcommand*{\fzsign}{\CJKfamily{fzsigns}}

%CC 协议声明用字(版权页上方 CC 图标) 可在 CC 网站获取
\newfontfamily\ccicons{CC Icons}

%------数学-----------------------------------------------------------------
\DeclareSymbolFont{euexlargesymbols}{U}{euex}{m}{n}%选择有直立积分号的AMS 设计的euex字体族
\DeclareMathSymbol{\intop}{\mathop}{euexlargesymbols}{"52}%得到直立的定积分和重积分号
\DeclareMathSymbol{\ointop}{\mathop}{euexlargesymbols}{"48}%得到直立的环路积分号

%方程式编号设置: 2-17
\renewcommand{\theequation}{\thechapter -\arabic{equation}}

%定理环境
\usepackage[framed,thmmarks,amsmath,thref]{ntheorem}	%自定义定理宏包
\theoremstyle{nonumberplain}
\theorempostskipamount=0em
\theorempreskipamount=0em
\theoremsymbol{\textcolor{gray}{$\blacksquare$}}
\theoremheaderfont{\normalfont\sffamily}
\theorembodyfont{\rmfamily}
\newtheorem{Definition}{\indent 定义}
\newtheorem{Theorem}[Definition]{\indent 规律}	%定理
\newtheorem{Example}{\indent 例题}[chapter]
\newtheorem{Proof}{\indent 证明}	%证明

%重新定义\eqref的英文括号为中文
\renewcommand{\eqref}[1]{(\ref{#1})}


%------图(表)题注------------------------------------------------------
\usepackage[font=small,
   justification=raggedright,
       labelfont=bf,
        labelsep=colon,
          margin=25pt,
   tableposition=top]{caption}

\DeclareCaptionFont{kai}{\kaishu}
\captionsetup{textfont=kai}

\usepackage{subcaption}

%------脚注 & 边注--------------------------------------------------------
% 脚注及其编号设置
\usepackage[perpage,stable,marginal,symbol*]{footmisc} %norule
\skip\footins=20pt plus 4pt minus 2pt

\usepackage{pifont}
\DefineFNsymbols{circled}{{\ding{172}}{\ding{173}}{\ding{174}}
{\ding{175}}{\ding{176}}{\ding{177}}{\ding{178}}{\ding{179}}{\ding{180}}{\ding{181}}}
\setfnsymbol{circled}

%边注/页边索引设置
\usepackage{marginnote}
\renewcommand*{\marginfont}{\color{red}\kaishu}

%------tikz------------------------------------------------------------
\newcommand*\tikzname{Ti\emph{k}Z} %只是定义TikZ这个文本

\usepackage{tikz}
\usetikzlibrary{backgrounds,
                calc,
                decorations.pathmorphing,
                er,
                petri,
                positioning,
                shapes.arrows}
                
\usepackage{pgfplots}
\usepgflibrary{decorations.pathmorphing}
\usepgfplotslibrary{patchplots}

%pgfplot通用绘图设定
\pgfplotsset{compat=1.5.1,
    every node/.style={font=\small\itshape},
    every axis/.append style={
        axis on top,black,thick,line join=round,line cap=round,
        width=200pt,height=200pt,
        font=\small\itshape,
        axis lines=middle,axis line style={-},
        extra x tick style={tickwidth=0},
		extra y tick style={tickwidth=0},
        xtick=\empty,ytick=\empty,
        xticklabel=\ ,yticklabel=\ ,
        area style,axis background/.style={fill=white}
    }
}

%------配色设置------------------------------------------------------------
%rgb (0~1) RGB (0~255)
\globalcolorstrue
\xglobal\definecolor{blueF}{RGB}{180,207,236}
\xglobal\definecolor{blueL}{RGB}{114,159,207}
\xglobal\definecolor{blue}{RGB}{32,74,135}
\xglobal\definecolor{ccgray}{RGB}{171,179,172}      %CC icons 官方指定配色
\xglobal\definecolor{greenD}{RGB}{0,104,56}
\xglobal\definecolor{greenF}{RGB}{177,194,147}
\xglobal\definecolor{greenL}{RGB}{138,226,52}
\xglobal\definecolor{green}{RGB}{78,154,6}
\xglobal\definecolor{orange}{RGB}{213,125,74}
\xglobal\definecolor{purpleD}{RGB}{115,55,110}
\xglobal\definecolor{purpleF}{RGB}{231,212,217}
\xglobal\definecolor{purpleL}{RGB}{203,158,171}
\xglobal\definecolor{purple}{RGB}{165,84,116}
\xglobal\definecolor{redF}{RGB}{233,200,184}
\xglobal\definecolor{redL}{RGB}{239,41,41}
\xglobal\definecolor{red}{RGB}{164,0,0}
\xglobal\definecolor{shadecolor}{RGB}{247,247,247}
\xglobal\definecolor{titleB}{RGB}{86,165,221}
\xglobal\definecolor{titleC}{RGB}{153,175,92}
\xglobal\definecolor{titleD}{RGB}{249,155,65}

%渐变色
\pgfplotsset{colormap={bluewhite}{color(0cm)=(blue); color(1cm)=(white);}}
\pgfplotsset{colormap={hot}{color(0cm)=(red); color(1cm)=(yellow); color(3cm)=(green)}}
\pgfplotsset{colormap={redyellow}{color(0cm)=(red); color(1cm)=(yellow);}}
\pgfplotsset{colormap={yellowgreen}{color(0cm)=(yellow); color(1cm)=(green);}}

%------章节标题------------------------------------------------------------
\CTEXsetup[name={第,部分},
     nameformat={\centering\itshape},
    titleformat={\Huge\itshape}]{part}							%部
\CTEXsetup[name={第,章},
     beforeskip={50pt plus 2pt minus 2pt},
         number={\arabic{chapter}},
     nameformat={\LARGE\sffamily},
    titleformat={\LARGE\sffamily},
   numberformat={\LARGE\sffamily},
      afterskip={20pt plus 2pt minus 2pt},
      aftername={\quad}]{chapter}								%章
\CTEXsetup[name={,},
	 beforeskip={16pt plus 2pt minus 2pt},
         format={\large\flushleft\sffamily},
      aftername={\quad},
      afterskip={10pt plus 2pt minus 2pt}]{section}			%节
\CTEXsetup[name={,.\enspace},
	 beforeskip={1pt},
         number={\arabic{subsection}},
        format+={\normalsize},
   %numberformat={\colorbox{blue!15}},
      aftername={},
      afterskip={1pt}]{subsection}								%小节
\CTEXsetup[name={,.},
         number={\arabic{subsubsection}},
         format={\zihao{5}}]{subsubsection}						%小小节
\CTEXsetup[name={附录},
         number={\arabic{chapter}}]{appendix}					%附录

%------目录 & 图表目录-------------------------------------------------------
\setcounter{secnumdepth}{4}

\usepackage{titletoc}
\titlecontents{figure}
	[2.5em]
	{}
	{\contentslabel{2.5em}}
	{}
	{\titlerule*[0.5pc]{$\cdot$}\small\rmfamily\contentspage}
\titlecontents{part}		            	%{section}
	[0ex]				                	%[left]
	{\vspace{3mm}\centering\sffamily}		%{above-code}
	{}		                                %{numbered-entry-format}
	{}						               	%{numberless-entry-format}
	{}						            	%{filler-page-format}
	%[\hfill]				            	%{below-code}
\titlecontents{chapter}                     %目录条目:章
	[4em]
	{\sffamily}		                        %{\vspace{3mm}\sffamily}
	{\contentslabel{4em}}
	{}
	{\titlerule*[0.5pc]{$\cdot$}\small\rmfamily\contentspage}
\titlecontents{section}                     %目录条目:节
	[4em]
	{}	                                    %{\small}
	{\contentslabel{2.5em}}
	{}
	{\titlerule*[0.5pc]{$\cdot$}\small\rmfamily\contentspage}

\titlecontents*{subsection}                 %目录条目:小节
[4em]
{\small\kaishu}
{}
{}
{(\thecontentspage)}
[ \textcolor{gray}{\textbullet}\ ]
[]

\setcounter{tocdepth}{3}

%设置图表目录
\newcommand*{\noaddvspace}{\renewcommand*{\addvspace}[1]{}}
\addtocontents{lof}{\protect\noaddvspace}
\addtocontents{lof}{\linespread{1.25}\selectfont}

%----------------------------------------------------------------------
\linespread{1.25}

\hfuzz3pt % Overfull Box < 3pt 不警告

%----------------------------------------------------------------------
\newcommand{\documentEdition}{0.1-r083}

%----------------------------------------------------------------------
\makeindex

\listfiles