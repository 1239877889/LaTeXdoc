\begin{tikzpicture}
%(x^0.5 + 1) (y^0.5 + 1) - 1
\begin{axis}[
domain=0:10,
axis on top=true,
view={-45}{45},%view={-25}{60},
xmin=0,xmax=10,ymin=0,ymax=10,zmin=0,zmax=18,
xtick=\empty,ytick=\empty,ztick=\empty,
xticklabel=\ ,yticklabel=\ ,zticklabel=\ ,
width=180pt,height=150pt,
scale only axis,
axis x line*=middle,
axis y line*=middle,
axis z line*=none,
axis line style={-},
extra x ticks={9},
extra x tick style={tickwidth=0},
extra x tick labels={$x$},
extra y ticks={9},
extra y tick style={tickwidth=0},
extra y tick labels={$y$},
extra description/.code={
	\node[below] at (axis cs:0,0,0) {$O$};
	\node[above] at (axis cs:10,10,17) {$z$};
},
axis background/.style={fill=black!3}]
\addplot3[fill=white,draw=none] coordinates {
(0,0,0)
(0,10,0)
(10,10,0)
(10,0,0)
};
%\addplot3 table[x=x,y=y,z=z] {pgfplots_scatterdata4.dat};

%\addplot3[left color=green,right color=green,middle color=yellow,draw=none]
\addplot3[fill=green,fill opacity=0.5,draw=none]
coordinates {
(2.29814,10.00001158,9.472)
(2.3,9.993596697,9.472)
(2.45,9.500635151,9.472)
(2.6,9.051357661,9.472)
(2.75,8.640028364,9.472)
(2.9,8.261899936,9.472)
(3.05,7.913005333,9.472)
(3.2,7.590000799,9.472)
(3.35,7.290045837,9.472)
(3.5,7.01071027,9.472)
(3.65,6.74990144,9.472)
(3.8,6.505806579,9.472)
(3.95,6.276846753,9.472)
(4.1,6.061639726,9.472)
(4.25,5.85896977,9.472)
(4.4,5.667762939,9.472)
(4.55,5.487066676,9.472)
(4.7,5.316032881,9.472)
(4.85,5.153903771,9.472)
(5,5,9.472)
(5.15,4.85371063,9.472)
(5.3,4.714484617,9.472)
(5.45,4.581823552,9.472)
(5.6,4.455275445,9.472)
(5.75,4.334429378,9.472)
(5.9,4.218910884,9.472)
(6.05,4.108377952,9.472)
(6.2,4.002517541,9.472)
(6.35,3.901042542,9.472)
(6.5,3.803689123,9.472)
(6.65,3.710214386,9.472)
(6.8,3.620394317,9.472)
(6.95,3.534021962,9.472)
(7.1,3.45090582,9.472)
(7.25,3.370868413,9.472)
(7.4,3.293745016,9.472)
(7.55,3.219382518,9.472)
(7.7,3.14763841,9.472)
(7.85,3.078379876,9.472)
(8,3.011482974,9.472)
(8.15,2.9468319,9.472)
(8.3,2.884318331,9.472)
(8.45,2.823840821,9.472)
(8.6,2.765304262,9.472)
(8.75,2.708619394,9.472)
(8.9,2.653702357,9.472)
(9.05,2.600474289,9.472)
(9.2,2.548860954,9.472)
(9.35,2.498792408,9.472)
(9.5,2.450202691,9.472)
(9.65,2.403029543,9.472)
(9.8,2.357214152,9.472)
(9.95,2.312700911,9.472)
(10,2.298143356,9.472)
(10,2.298143356,0)
(9.95,2.312700911,0)
(9.8,2.357214152,0)
(9.65,2.403029543,0)
(9.5,2.450202691,0)
(9.35,2.498792408,0)
(9.2,2.548860954,0)
(9.05,2.600474289,0)
(8.9,2.653702357,0)
(8.75,2.708619394,0)
(8.6,2.765304262,0)
(8.45,2.823840821,0)
(8.3,2.884318331,0)
(8.15,2.9468319,0)
(8,3.011482974,0)
(7.85,3.078379876,0)
(7.7,3.14763841,0)
(7.55,3.219382518,0)
(7.4,3.293745016,0)
(7.25,3.370868413,0)
(7.1,3.45090582,0)
(6.95,3.534021962,0)
(6.8,3.620394317,0)
(6.65,3.710214386,0)
(6.5,3.803689123,0)
(6.35,3.901042542,0)
(6.2,4.002517541,0)
(6.05,4.108377952,0)
(5.9,4.218910884,0)
(5.75,4.334429378,0)
(5.6,4.455275445,0)
(5.45,4.581823552,0)
(5.3,4.714484617,0)
(5.15,4.85371063,0)
(5,5,0)
(4.85,5.153903771,0)
(4.7,5.316032881,0)
(4.55,5.487066676,0)
(4.4,5.667762939,0)
(4.25,5.85896977,0)
(4.1,6.061639726,0)
(3.95,6.276846753,0)
(3.8,6.505806579,0)
(3.65,6.74990144,0)
(3.5,7.01071027,0)
(3.35,7.290045837,0)
(3.2,7.590000799,0)
(3.05,7.913005333,0)
(2.9,8.261899936,0)
(2.75,8.640028364,0)
(2.6,9.051357661,0)
(2.45,9.500635151,0)
(2.3,9.993596697,0)
(2.29814,10.00001158,0)
};

%----------------------------------------------------------------------
%补偿z轴
%\addplot3[black] coordinates {
%(10,10,16.325)
%(10,10,18)
%};
%补偿坐标系边框
\addplot3[gray,dashed,thin] coordinates {
(0,10,0)
(10,10,0)
(10,0,0)
};
%补偿z轴
\addplot3[gray,dashed,thin] coordinates {
(10,10,0)
(10,10,16.324)
};
%补偿xymax的纵向边界
\addplot3[gray,dashed,thin] coordinates {
(0,10,0)
(0,10,3.163)
};
\addplot3[gray,dashed,thin] coordinates {
(10,0,0)
(10,0,3.163)
};

\addplot3 graphics[
points={%important
(10,0,3.16) => (600,180)
(8,8,8) => (301,318)
(10,10,16.325) => (301,464)
(0,0,0) => (301,1)
}] {qumianq.png};

%----------------------------------------------------------------------
%z=z(2.5,2.5)=5.66228 的无差异曲线
%x=0.36076, y=10
%x=10, y=0.36076
\addplot3[draw=green,ultra thick,domain=0.36076:10,samples=40,samples y=0] ({x},{((2.5^0.5+1)^2/(x^0.5 + 1) - 1)^2},{5.66228}); 
\addplot3[draw=gray,domain=0.36076:10,samples=40,samples y=0,dashed] ({x},{((2.5^0.5+1)^2/(x^0.5 + 1) - 1)^2},{0}); 
\addplot3[gray,dashed,thin] coordinates {
(10,0.36076,0)
(10,0.36076,5.66228)
};
\addplot3[gray,dashed,thin] coordinates {
(0.36076,10,0)
(0.36076,10,5.66228)
};
\node[above] at (axis cs:0.36076,10,5.66228) {$A$};
\node[above] at (axis cs:10,0.36076,5.66228) {$B$};

%----------------------------------------------------------------------
%z=z(5,5)=9.472 的无差异曲线
%x=2.29814, y=10
%x=10, y=2.29814
\addplot3[draw=green,ultra thick,domain=2.29814:10,samples=40,samples y=0] ({x},{((5^0.5+1)^2/(x^0.5 + 1) - 1)^2},{9.472}); 
\addplot3[draw=gray,domain=2.29814:10,samples=40,samples y=0,dashed] ({x},{((5^0.5+1)^2/(x^0.5 + 1) - 1)^2},{0}); 
\addplot3[gray,dashed,thin] coordinates {
(10,2.29814,0)
(10,2.29814,9.472)
};
\addplot3[gray,dashed,thin] coordinates {
(2.29814,10,0)
(2.29814,10,9.472)
};
\node[above] at (axis cs:2.29814,10,9.472) {$C$};
\node[above] at (axis cs:10,2.29814,9.472) {$D$};
\node[below] at (axis cs:2.29814,10,0) {$C'$};
\node[below] at (axis cs:10,2.29814,0) {$D'$};
%----------------------------------------------------------------------
% z = z(7.5,7.5) = 12.9772 的无差异曲线
% x = 5.5605, y = 10
% x = 10, y = 5.5605
\addplot3[draw=green,ultra thick,domain=5.5605:10,samples=40,samples y=0] ({x},{((7.5^0.5+1)^2/(x^0.5 + 1) - 1)^2},{12.9772}); 
\addplot3[draw=gray,domain=5.5605:10,samples=40,samples y=0,dashed] ({x},{((7.5^0.5+1)^2/(x^0.5 + 1) - 1)^2},{0}); 
\addplot3[gray,dashed,thin] coordinates {
(10,5.5605,0)
(10,5.5605,12.9772)
};
\addplot3[gray,dashed,thin] coordinates {
(5.5605,10,0)
(5.5605,10,12.9772)
};
\node[above] at (axis cs:5.5605,10,12.9772) {$E$};
\node[above] at (axis cs:10,5.5605,12.9772) {$F$};

%----------------------------------------------------------------------
%灰色的边界截面
%\addplot3[draw=none,fill=gray,fill opacity=0.75,samples=40,samples y=0] ({x},{0},{sqrt(x)}) \closedcycle;
\addplot3[draw=none,fill=gray,fill opacity=0.75,samples=40,samples y=0] ({x},{0},{sqrt(x)}) --cycle;
\addplot3[fill=gray,fill opacity=0.75,draw=none] coordinates {
(0,0,0)
(10,0,0)
(10,0,3.163)
};

\addplot3[draw=none,fill=gray,fill opacity=0.75,samples=40] ({0},{y},{sqrt(y)}) --cycle;
\addplot3[fill=gray,fill opacity=0.75,draw=none] coordinates {
(0,0,0)
(0,10,0)
(0,10,3.163)
};

%----------------------------------------------------------------------
%绘制曲面与坐标系边界的交线
%(x^0.5 + 1) (y^0.5 + 1) - 1
\addplot3[draw=red,thick,samples=40,samples y=0] ({x},{0},{sqrt(x)}); %samples=100,samples y=0
\addplot3[draw=red,thick,samples=60,samples y=0] ({x},{10},{4.16*(sqrt(x)+1)-1});
\addplot3[draw=blue,thick,samples=40] ({0},{y},{sqrt(y)});
\addplot3[draw=blue,thick,samples=40] ({10},{y},{4.16*(sqrt(y)+1)-1});
\end{axis}
\end{tikzpicture}